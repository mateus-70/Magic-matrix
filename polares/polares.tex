\documentclass{article}
\usepackage{amsmath}
\usepackage{amssymb}
\usepackage{amsthm}
%\usepackage[portuges]{babel}
\usepackage[utf8]{inputenc}
\usepackage[T1]{fontenc}
\usepackage{enumerate}

\let\oldln\ln
\renewcommand{\ln}[1]{ \oldln {(#1)}}
\newcommand{\abs}[1]{\lvert #1 \rvert}

%\let\oldint\int
%\newcommand{\fint}[3][s][s]{ \int_{#1}^{#2}{\left(#3\right)}}

%\newcommand{\dint}[4]{ \int_{#1}^{#2} {\left(#3\right)}\, \mathrm{d}#4}

\newcommand{\paren}[1]{ \left( #1 \right)}
\newcommand{\colch}[1]{ \left[ #1 \right]}
\newcommand{\tr}[1]{\mathrm{tr} \left( #1 \right)}
\let\olddet\det
\renewcommand{\det}[1]{\olddet{ \left( #1 \right)} }

\providecommand{\sen}{\mathrm{--ERRO--}}
\providecommand{\senh}{\mathrm{--ERRO--}}
\providecommand{\arcsen}{\mathrm{--ERRO--}}
\providecommand{\arsenh}{\mathrm{--ERRO--}}

\renewcommand{\sen}[1]{\mathrm{sen}\, \left( #1 \right)}
\renewcommand{\arcsen}[1]{\mathrm{arcsen}\, \left( #1 \right)}
\renewcommand{\senh}[1]{\mathrm{senh}\, \left( #1 \right)}
\renewcommand{\arsenh}[1]{\mathrm{arsenh}\, \left( #1 \right)}

\providecommand{\cos}{\mathrm{--ERRO--}}
\providecommand{\cosh}{\mathrm{--ERRO--}}
\providecommand{\arccos}{\mathrm{--ERRO--}}
\providecommand{\arcosh}{\mathrm{--ERRO--}}

\renewcommand{\cos}[1]{\mathrm{cos}\, \left( #1 \right)}
\renewcommand{\cosh}[1]{\mathrm{cosh}\, \left( #1 \right)}
\renewcommand{\arccos}[1]{\mathrm{arccos}\, \left( #1 \right)}
\renewcommand{\arcosh}[1]{\mathrm{arcosh}\, \left( #1 \right)}

\providecommand{\tg}{\mathrm{--ERRO--}}
\providecommand{\tgh}{\mathrm{--ERRO--}}
\providecommand{\arctg}{\mathrm{--ERRO--}}
\providecommand{\artgh}{\mathrm{--ERRO--}}

\renewcommand{\tg}[1]{\mathrm{tg}\,\left( #1 \right)}
\renewcommand{\tgh}[1]{\mathrm{tgh}\, \left( #1 \right)}
\renewcommand{\arctg}[1]{\mathrm{arctg}\, \left( #1 \right)}
\renewcommand{\artgh}[1]{\mathrm{artg}\, \left( #1 \right)}

\providecommand{\sec}{\mathrm{--ERRO--}}
\providecommand{\cossec}{\mathrm{--ERRO--}}
\providecommand{\cotg}{\mathrm{--ERRO--}}

\renewcommand{\sec}[1]{\mathrm{sec}\, \left( #1 \right)}
\renewcommand{\cossec}[1]{\mathrm{cossec}\, \left( #1 \right)}
\renewcommand{\cotg}[1]{\mathrm{cotg}\, \left( #1 \right)}

\providecommand{\arcsec}{\mathrm{--ERRO--}}
\providecommand{\arccossec}{\mathrm{--ERRO--}}
\providecommand{\arccotg}{\mathrm{--ERRO--}}

\renewcommand{\arcsec}[1]{\mathrm{sec}\, \left( #1 \right)}
\renewcommand{\arccossec}[1]{\mathrm{cossec}\, \left( #1 \right)}
\renewcommand{\arccotg}[1]{\mathrm{cotg}\, \left( #1 \right)}

\providecommand{\sech}{\mathrm{--ERRO--}}
\providecommand{\cossech}{\mathrm{--ERRO--}}
\providecommand{\cotgh}{\mathrm{--ERRO--}}

\renewcommand{\sech}[1]{\mathrm{sech}\, \left( #1 \right)}
\renewcommand{\cossech}[1]{\mathrm{cossech}\, \left( #1 \right)}
\renewcommand{\cotgh}[1]{\mathrm{cotgh}\, \left( #1 \right)}

\providecommand{\arsech}{\mathrm{--ERRO--}}
\providecommand{\arcossech}{\mathrm{--ERRO--}}
\providecommand{\arcotgh}{\mathrm{--ERRO--}}

\renewcommand{\arsech}[1]{\mathrm{arsech}\, \left( #1 \right)}
\renewcommand{\arcossech}[1]{\mathrm{arcossech}\, \left( #1 \right)}
\renewcommand{\arcotgh}[1]{\mathrm{arcotgh}\, \left( #1 \right)}


\begin{document}
    %$r = \mathrm{d}(O,P)$
    %$\theta = \mathrm{ang}(
    Em termos analíticos $r \ge 0$.
    Curvas típicas polares \\

       Círculo raio $a$: $r = a$ \\
       $$
            r = a \cdot \cos \theta \, \mathrm{ou} \, r = a \cdot \sen \theta
       $$
       
       
       Rosáceas: 
       \begin{equation*}
           r = a \cdot \cos {n \theta}\, \mathrm{ ou} \,
           r = a \cdot \sen {n \theta}
       \end{equation*}
       Obs.: Se $n$ é par, a rosácea é desenhada em $ [ 0, 2 \pi ]$
             Se $n$ é ímpar, a rosácea é desenhada em $ [ 0, \pi ]$
       
       
       Cardióide: 
       \begin{equation*}
            r = a + a \cdot \cos {\theta}\, \mathrm{ou}\, r = a + a \cdot \sen \theta
       \end{equation*}

       
       Lemniscata: 
       \begin{equation*}
           r^2 = a^2 \cdot \sen {2\theta} \implies \abs{r} = a \sqrt{ \sen{2\theta} } \, \mathrm{ou} \, r^2 = a^2 \cdot \cos{2\theta} \implies \abs{r} = a \sqrt{ \cos{2\theta} } \\
       \end{equation*}
       Obs.: Cuidado com o domínio: 
       \begin{align*}
            & \abs{r} = a \sqrt{ \sen {2 \theta} } \\
            & \iff \sen {2 \theta} \ge 0  \\
            & \iff 2 \theta \in [0, \pi] \\
            & \iff \theta \in [0, \frac{\pi}{2} ] \\
       \end{align*}
       
       
       Limaçons: \\
       $$
        r = a + b \cdot \sen{\theta} \, \mathrm{ou} \,
        r = a + b \cdot \cos{\theta} 
        $$
       Obs.:
       \begin{enumerate}[i]
           \item Se $\abs{a} = \abs{b}$ temos o cardióide
           \item Se $\abs{a} < \abs{b}$ limaçon com laço
           \item Se $\abs{a} > \abs{b}$ limaçon sem laço
       \end{enumerate}
\end{document}
