\documentclass[english,ngerman,parskip=half]{scrartcl}
\usepackage{amsmath}
\usepackage{amssymb}
\usepackage{amsthm}
%\usepackage[portuges]{babel}
\usepackage[utf8]{inputenc}
\usepackage{graphicx}
\usepackage[T1]{fontenc}
\usepackage{systeme}
\usepackage{yfonts}
\newcommand{\abs}[1]{\lvert #1 \rvert}

%\usepackage{libertine}
\usepackage{microtype}
\usepackage{lmodern}
\usepackage[brazilian]{babel}
\usepackage{xcolor}

\setcounter{MaxMatrixCols}{20}
\usepackage{tikz}
%\pgfplotsset{compat=newest}
\usetikzlibrary{shapes,positioning,intersections,quotes}

\theoremstyle{definition}
\newtheorem{definition}{Definição}
\newtheorem{theorem}{Teorema}
\newtheorem*{remark}{Lembrete}

\pagestyle{empty}

\begin{document}
\textbf{Anexo 2} - Definições e teoremas
\begin{definition}{Círculo unitário}
       no plano cartesiano xy, é o círculo de raio 1 cujo centro é a origem O(0,0).
\end{definition}
\begin{definition}{Valor do cos de um ângulo t}
        é a primeira coordenada do ponto projetado sobre o eixo x (também chamado de eixo dos cossenos).
\end{definition}
\begin{definition}{Valor do sen de um ângulo t}
    é o valor da segunda coordenada do ponto projetado sobre o eixo y (ou eixo dos senos).
\end{definition}
\begin{theorem}
    $\forall t \in \mathbb{R}$, vale: $\sin^2t + \cos^2t = 1$. (Tácito que é o círculo unitário)
\end{theorem}
\begin{theorem}
    Sejam $P=(a,b)$ e $Q=(c,d)$ pontos de $\mathbb{R}^2$, a distância entre eles, $d(P,Q)$ é dada por: 
    $\sqrt{(a-c)^2 + (b-d)^2}$
\end{theorem}
\begin{remark}
       Note que $(\cos t, \sin t)$ é a coordenada do ponto sobre a circunferência cujo ângulo com o $eixo\ x$ é $t$.
\end{remark}
%\begin{tikzpicture}
%    \draw[step=0.2cm,gray,very thin] (-2,-2) grid (6,6);
%    \draw (0,2) circle (3cm);
%\end{tikzpicture}
\end{document}

