\documentclass[english,ngerman,parskip=half]{scrartcl}
\usepackage{amsmath}
\usepackage{amssymb}
\usepackage{amsthm}
%\usepackage[portuges]{babel}
\usepackage[utf8]{inputenc}
\usepackage{graphicx}
\usepackage[T1]{fontenc}
\usepackage{systeme}
\usepackage{yfonts}
\newcommand{\abs}[1]{\lvert #1 \rvert}

%\usepackage{libertine}
\usepackage{microtype}
\usepackage{lmodern}
\usepackage[brazilian]{babel}
\usepackage{xcolor}

\setcounter{MaxMatrixCols}{20}
\usepackage{tikz}
%\pgfplotsset{compat=newest}
\usetikzlibrary{shapes,positioning,intersections,quotes}

\usepackage{enumitem}
\usepackage{mathtools}
%\newcommand{\mymbox}[1][def]{\mbox}
%\newcommand\myeq{\stackrel{\mathclap{\normalfont\mymbox}}{=}}
\pagestyle{empty}

\begin{document}
\textbf{Anexo 3} - FOLHA DO PROFESSOR \\
Sejam $s$ e $t$ ângulos, $c$ o círculo unitário. Seja $P(1,0)$ um ponto. 
\begin{enumerate}[label=\textnormal{(\arabic*)}]

    \item Escreva as coordenadas $(x,y)$ do ponto de $c$ (em função de $s$ ou $t$) quando os ângulos são respectivamente $s$ e $t$.\\
        $T = (\cos t, \sin t)$ e $S = (\cos s, \sin s)$
    \item Calcule a distância entre os pontos encontrados anteriormente pela fórmula da distância entre pontos.
        $$d\Big(T,S\Big) = \sqrt{(\cos t - \cos s)^2 + (\sin t - \sin s)^2} =
        \sqrt{2 - \cos t \cdot \cos s - \sin t \cdot \sin s}
        $$
    \item Quanto mede a corda que liga $P$ à qualquer ponto da circunferência correspondente a um ângulo $x$? E em particular para $t-s$?
        \begin{align*}
             d \Big( (1,0), (T-S) \Big) &= \sqrt{ (\cos(t-s)-1)^2 + \sin^2(t-s) } \\
                                      &= \sqrt{\cos^2(t-s) - 2\cdot \cos(t-s) + 1 + \sin^2(t-s)} \\
                                      &= \sqrt{2(1 - \cos(t-s)} 
        \end{align*}
    \item Essas cordas têm o mesmo tamanho? Por quê? \\
        Sim, pois subtendem arcos de mesmo tamanho.
    \item Chegaste então em: $\cos (t-s) = \cos t \cdot \cos s + \sin t \cdot \sin s$
    \item Que fórmulas podemos obter fixando t em valores nos eixos (tente $0$, $\pi/2$, $-\pi/2$)?
        \begin{enumerate}
            \item \label{equacao} Se $t = 0 \rightarrow \cos (0-s) = \cos 0 \cdot \cos s + \sin 0 \cdot \sin s \implies \cos -s = \cos s$
            \item Se $t = \frac{\pi}{2} \rightarrow \cos (\frac{\pi}{2}-s) = \cos \frac{\pi}{2} \cdot \cos s +
                \sin \left(\frac{\pi}{2}\right) \cdot \sin s \implies \cos \left(\frac{\pi}{2} - s\right) = \sin s$
            \item Se $t = \frac{-\pi}{2} \\ \rightarrow \cos (\frac{-\pi}{2}-s) = \cos \frac{-\pi}{2} \cdot \cos s +
                \sin \left( \frac{-\pi}{2} \right) \cdot \sin s \implies \cos \left(\dfrac{-\pi}{2} - s\right) = -\sin s$
        \end{enumerate}
    \item O que podemos afirmar sobre as funções seno, cosseno a respeito de serem pares ou ímpares?
        \begin{enumerate}
            \item $cos$ é par, pela equação \ref{equacao}
            \item Se $x = -s$, temos \\
                $\sin -s = \sin x = \cos( \frac{\pi}{2} - x) = \cos (\frac{\pi}{2} + s) = \cos (\frac{-\pi}{2} - s) = -\sin s $
        \end{enumerate}
    \item Quanto é $\sin(t-s)$?
        \begin{align}
            \sin (a - b) &=- \cos\left(\dfrac{\pi}{2} + ( a - b )\right) \\
                         &=- \left[ \cos\left(\dfrac{\pi}{2} + a\right) \cdot \cos b + \sin\left(\dfrac{\pi}{2} + a\right) \cdot \sin b \right] \\
                         &= \sin a \cdot \cos b - \sin b \cdot \cos a
        \end{align}
    %\item E $\sin(t+s)$? 
\end{enumerate}
\end{document}

