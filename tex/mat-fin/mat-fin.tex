\documentclass{book}
\usepackage[utf8]{inputenc}
\usepackage{amsmath}
\usepackage{amssymb}
\usepackage{amsthm}
%\usepackage[portuges]{babel}
\usepackage[utf8]{inputenc}
\usepackage[T1]{fontenc}

%\let\oldln\ln
\renewcommand{\ln}[1]{ \oldln {(#1)}}
\newcommand{\abs}[1]{\lvert #1 \rvert}

%\let\oldint\int
%\newcommand{\fint}[3][s][s]{ \int_{#1}^{#2}{\left(#3\right)}}

%\newcommand{\dint}[4]{ \int_{#1}^{#2} {\left(#3\right)}\, \mathrm{d}#4}

\newcommand{\paren}[1]{ \left( #1 \right)}
\newcommand{\colch}[1]{ \left[ #1 \right]}
\newcommand{\tr}[1]{\mathrm{tr} \left( #1 \right)}
\let\olddet\det
\renewcommand{\det}[1]{\olddet{ \left( #1 \right)} }

%\providecommand{\sen}{\mathrm{--ERRO--}}
\providecommand{\senh}{\mathrm{--ERRO--}}
\providecommand{\arcsen}{\mathrm{--ERRO--}}
\providecommand{\arsenh}{\mathrm{--ERRO--}}

\renewcommand{\sen}[1]{\mathrm{sen}\, \left( #1 \right)}
\renewcommand{\arcsen}[1]{\mathrm{arcsen}\, \left( #1 \right)}
\renewcommand{\senh}[1]{\mathrm{senh}\, \left( #1 \right)}
\renewcommand{\arsenh}[1]{\mathrm{arsenh}\, \left( #1 \right)}

\providecommand{\cos}{\mathrm{--ERRO--}}
\providecommand{\cosh}{\mathrm{--ERRO--}}
\providecommand{\arccos}{\mathrm{--ERRO--}}
\providecommand{\arcosh}{\mathrm{--ERRO--}}

\renewcommand{\cos}[1]{\mathrm{cos}\, \left( #1 \right)}
\renewcommand{\cosh}[1]{\mathrm{cosh}\, \left( #1 \right)}
\renewcommand{\arccos}[1]{\mathrm{arccos}\, \left( #1 \right)}
\renewcommand{\arcosh}[1]{\mathrm{arcosh}\, \left( #1 \right)}

\providecommand{\tg}{\mathrm{--ERRO--}}
\providecommand{\tgh}{\mathrm{--ERRO--}}
\providecommand{\arctg}{\mathrm{--ERRO--}}
\providecommand{\artgh}{\mathrm{--ERRO--}}

\renewcommand{\tg}[1]{\mathrm{tg}\,\left( #1 \right)}
\renewcommand{\tgh}[1]{\mathrm{tgh}\, \left( #1 \right)}
\renewcommand{\arctg}[1]{\mathrm{arctg}\, \left( #1 \right)}
\renewcommand{\artgh}[1]{\mathrm{artg}\, \left( #1 \right)}

\providecommand{\sec}{\mathrm{--ERRO--}}
\providecommand{\cossec}{\mathrm{--ERRO--}}
\providecommand{\cotg}{\mathrm{--ERRO--}}

\renewcommand{\sec}[1]{\mathrm{sec}\, \left( #1 \right)}
\renewcommand{\cossec}[1]{\mathrm{cossec}\, \left( #1 \right)}
\renewcommand{\cotg}[1]{\mathrm{cotg}\, \left( #1 \right)}

\providecommand{\arcsec}{\mathrm{--ERRO--}}
\providecommand{\arccossec}{\mathrm{--ERRO--}}
\providecommand{\arccotg}{\mathrm{--ERRO--}}

\renewcommand{\arcsec}[1]{\mathrm{sec}\, \left( #1 \right)}
\renewcommand{\arccossec}[1]{\mathrm{cossec}\, \left( #1 \right)}
\renewcommand{\arccotg}[1]{\mathrm{cotg}\, \left( #1 \right)}

\providecommand{\sech}{\mathrm{--ERRO--}}
\providecommand{\cossech}{\mathrm{--ERRO--}}
\providecommand{\cotgh}{\mathrm{--ERRO--}}

\renewcommand{\sech}[1]{\mathrm{sech}\, \left( #1 \right)}
\renewcommand{\cossech}[1]{\mathrm{cossech}\, \left( #1 \right)}
\renewcommand{\cotgh}[1]{\mathrm{cotgh}\, \left( #1 \right)}

\providecommand{\arsech}{\mathrm{--ERRO--}}
\providecommand{\arcossech}{\mathrm{--ERRO--}}
\providecommand{\arcotgh}{\mathrm{--ERRO--}}

\renewcommand{\arsech}[1]{\mathrm{arsech}\, \left( #1 \right)}
\renewcommand{\arcossech}[1]{\mathrm{arcossech}\, \left( #1 \right)}
\renewcommand{\arcotgh}[1]{\mathrm{arcotgh}\, \left( #1 \right)}


% plain definition remark em theoremstyle
    \theoremstyle{definition}
    \newtheorem{definition}{Definição}

    \theoremstyle{remark}
    \newtheorem*{remark}{Nota}

    \theoremstyle{plain}
    \newtheorem{theorem}{Teorema}

    \theoremstyle{plain}
    \newtheorem{corolary}{Corolário}

    \theoremstyle{plain}
    \newtheorem{lemma}{Lema}

\begin{document}
    Juros simples
    Para o cálculo de juros simples com uma iteração de tempo usamos a fórmula:
    $$J = P \cdot i$$, onde \\
    J: é o juros, o valor que o dinheiro aplicado "rendeu".
    P: é o dinheiro aplicado
    i: é a taxa de juros (no mesmo tempo de iteração da aplicação).
    Exemplo: Aplicar um capital de $R\$ 1000 $ com uma taxa de juros de $1\%$ ao mês (ou a.m.) por $1$ mês.
    Exemplo: Aplicar um capital de $R\$ 1000 $ com uma taxa de juros de $10\%$ ao ano (ou a.a.) por $1$ ano.
    
    Para calcular quanto de juros é gerado por $n$ iterações 
    de tempo modificamos a fórmula e ficamos com:
    $$J = P \cdot i \cdot n$$
    Uma característica dos juros simples é que cada iteração o valor gerado de juros é o mesmo. 
    Consideremos o caso a seguir.

    Exemplo: Aplicar um capital de $R\$ 1000 $ com uma taxa de juros de $10\%$aa por $5$ anos.
    No primeiro ano temos $J_1 = P \cdot i \cdot n = R\$ 1000 \cdot 0,10 = R\$ 100$ \\
    No segundo ano temos $J_2 = P \cdot i \cdot n = R\$ 1000 \cdot 0,10 = R\$ 100$ \\
    No terceiro ano temos $J_3 = P \cdot i \cdot n = R\$ 1000 \cdot 0,10 = R\$ 100$ \\
    No quarto ano temos $J_4 = P \cdot i \cdot n = R\$ 1000 \cdot 0,10 = R\$ 100$ \\
    No quinto ano temos $J_5 = P \cdot i \cdot n = R\$ 1000 \cdot 0,10 = R\$ 100$ \\
    Onde $J_k$ representa o juro gerado na iteração $k$ (neste caso em cada ano).

    Após a aplicação do capital, no ato da retirada não retiramos
    somente o juros gerado, mas também o principal aplicado.
    \begin{example}
        Ao aplicar $R\$ 500$ com taxa de $1\%$ am, por $12$ meses temos $J = R\$ 500 \cdot 0,01 \cdot 12 = R\$ 60 $. \\
        Complicado ao final do período poder retirar somente $R\$ 60$, não? Mas pode ser 
        retirado também o valor aplicado, neste caso $BRL 500$.
    \end{example}
    \begin{definition}{Montante}
        É o valor retirado no vencimento da aplicação. É calculado pela fórmula: $M = J + P$.
    \end{definition}
    \begin{definition}{Vencimento}
        É a data mínima que o montante poderá ser retirado.
    \end{definition}
    Isso quer dizer que se uma aplicação tem vencimento em $1$ ano, ele não poderá^1 ser retirado antes disso. \\
    ^1: Varia de acordo com a aplicação escolhida. Algumas têm período de carência, outras deixam retirar 
    parte do valor que seria obtido no vencimento. Alguns títulos podem ser vendidos no mercado secundário.
    
    $$This was intended to have a lot of exercises! LMFAO $$

    %% NEW CHAPTER %%

    Juro composto
    Nos juros simples cada iteração gera o mesmo valor que todas as outras. 
    No juro composto isso não acontece, porque cada iteração é calculada com o montante da iteração anterior 
    (exeto na primeiríssima vez que é calculado imediatamente com o principal).
    \begin{example}
        Vejamos uma tabela com o juro gerado por uma aplicação de $BRL 1000$ com uma taxa de $10\%am$ por $12$ meses.
    \end{example}
    Como é possível notar, cada nova iteração gera um juro levemente superior à iteração imediatamente anterior.
    Desta forma, calculamos o juro composto assim:
    $M_1 = P + P \cdot i = P(1 + i)$ \\
    $M_2 = M_1 + M_1 \cdot i = M_1 (1 + i) = P(1+i)(1+i)$ \\
    $M_3 = M_2 + M_2 \cdot i = M_2 (1 + i) = P(1+i)(1+i)(1+i)$ \\
    ...
    $M_n = M_{n-1} + M_{n-1} \cdot i = M_{n-1}(1 + i) = P(1+i)^n$ \\
    Portanto, para calcular quanto é o montante de uma aplicação usamos a fórmula: $M = P (1+i)^n$
    
\end{document}
