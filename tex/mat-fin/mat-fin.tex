\documentclass{book}
\usepackage[utf8]{inputenc}
\usepackage{amsmath}
\usepackage{amssymb}
\usepackage{amsthm}
\usepackage{enumerate}
\usepackage{appendix}
%\usepackage[portuges]{babel}
\usepackage[utf8]{inputenc}
\usepackage[T1]{fontenc}
\usepackage{siunitx}
\usepackage[lastexercise]{exercise}
\sisetup{group-separator={.}}
\sisetup{output-decimal-marker={,}}
\sisetup{group-minimum-digits=4}
%\let\oldln\ln
\renewcommand{\ln}[1]{ \oldln {(#1)}}
\newcommand{\abs}[1]{\lvert #1 \rvert}

%\let\oldint\int
%\newcommand{\fint}[3][s][s]{ \int_{#1}^{#2}{\left(#3\right)}}

%\newcommand{\dint}[4]{ \int_{#1}^{#2} {\left(#3\right)}\, \mathrm{d}#4}

\newcommand{\paren}[1]{ \left( #1 \right)}
\newcommand{\colch}[1]{ \left[ #1 \right]}
\newcommand{\tr}[1]{\mathrm{tr} \left( #1 \right)}
\let\olddet\det
\renewcommand{\det}[1]{\olddet{ \left( #1 \right)} }

%\providecommand{\sen}{\mathrm{--ERRO--}}
\providecommand{\senh}{\mathrm{--ERRO--}}
\providecommand{\arcsen}{\mathrm{--ERRO--}}
\providecommand{\arsenh}{\mathrm{--ERRO--}}

\renewcommand{\sen}[1]{\mathrm{sen}\, \left( #1 \right)}
\renewcommand{\arcsen}[1]{\mathrm{arcsen}\, \left( #1 \right)}
\renewcommand{\senh}[1]{\mathrm{senh}\, \left( #1 \right)}
\renewcommand{\arsenh}[1]{\mathrm{arsenh}\, \left( #1 \right)}

\providecommand{\cos}{\mathrm{--ERRO--}}
\providecommand{\cosh}{\mathrm{--ERRO--}}
\providecommand{\arccos}{\mathrm{--ERRO--}}
\providecommand{\arcosh}{\mathrm{--ERRO--}}

\renewcommand{\cos}[1]{\mathrm{cos}\, \left( #1 \right)}
\renewcommand{\cosh}[1]{\mathrm{cosh}\, \left( #1 \right)}
\renewcommand{\arccos}[1]{\mathrm{arccos}\, \left( #1 \right)}
\renewcommand{\arcosh}[1]{\mathrm{arcosh}\, \left( #1 \right)}

\providecommand{\tg}{\mathrm{--ERRO--}}
\providecommand{\tgh}{\mathrm{--ERRO--}}
\providecommand{\arctg}{\mathrm{--ERRO--}}
\providecommand{\artgh}{\mathrm{--ERRO--}}

\renewcommand{\tg}[1]{\mathrm{tg}\,\left( #1 \right)}
\renewcommand{\tgh}[1]{\mathrm{tgh}\, \left( #1 \right)}
\renewcommand{\arctg}[1]{\mathrm{arctg}\, \left( #1 \right)}
\renewcommand{\artgh}[1]{\mathrm{artg}\, \left( #1 \right)}

\providecommand{\sec}{\mathrm{--ERRO--}}
\providecommand{\cossec}{\mathrm{--ERRO--}}
\providecommand{\cotg}{\mathrm{--ERRO--}}

\renewcommand{\sec}[1]{\mathrm{sec}\, \left( #1 \right)}
\renewcommand{\cossec}[1]{\mathrm{cossec}\, \left( #1 \right)}
\renewcommand{\cotg}[1]{\mathrm{cotg}\, \left( #1 \right)}

\providecommand{\arcsec}{\mathrm{--ERRO--}}
\providecommand{\arccossec}{\mathrm{--ERRO--}}
\providecommand{\arccotg}{\mathrm{--ERRO--}}

\renewcommand{\arcsec}[1]{\mathrm{sec}\, \left( #1 \right)}
\renewcommand{\arccossec}[1]{\mathrm{cossec}\, \left( #1 \right)}
\renewcommand{\arccotg}[1]{\mathrm{cotg}\, \left( #1 \right)}

\providecommand{\sech}{\mathrm{--ERRO--}}
\providecommand{\cossech}{\mathrm{--ERRO--}}
\providecommand{\cotgh}{\mathrm{--ERRO--}}

\renewcommand{\sech}[1]{\mathrm{sech}\, \left( #1 \right)}
\renewcommand{\cossech}[1]{\mathrm{cossech}\, \left( #1 \right)}
\renewcommand{\cotgh}[1]{\mathrm{cotgh}\, \left( #1 \right)}

\providecommand{\arsech}{\mathrm{--ERRO--}}
\providecommand{\arcossech}{\mathrm{--ERRO--}}
\providecommand{\arcotgh}{\mathrm{--ERRO--}}

\renewcommand{\arsech}[1]{\mathrm{arsech}\, \left( #1 \right)}
\renewcommand{\arcossech}[1]{\mathrm{arcossech}\, \left( #1 \right)}
\renewcommand{\arcotgh}[1]{\mathrm{arcotgh}\, \left( #1 \right)}


% plain definition remark em theoremstyle
    \theoremstyle{definition}
    \newtheorem{definition}{Definição}

    \theoremstyle{remark}
    \newtheorem*{remark}{Nota}

    \theoremstyle{plain}
    \newtheorem{theorem}{Teorema}

    \theoremstyle{plain}
    \newtheorem{corolary}{Corolário}

    \theoremstyle{plain}
    \newtheorem{lemma}{Lema}

    \theoremstyle{plain}
    \newtheorem{example}{Exemplo}

    \title{Matemática Financeira}
    \author{Mateus Schroeder da Silva}
    \date{\today}

\begin{document}
    \maketitle
    \chapter{Juro simples}
    \begin{definition}[Juro]
        Juro é o valor que uma quantia em dinheiro "rende", se aplicada. \\
        É denotado pela letra $J$.
    \end{definition}
    \begin{definition}[Capital]
        O capital é o dinheiro que alguém tem a propriedade.
        É denotado pela letra $C$.
    \end{definition}
    \begin{definition}[Principal]
        O mesmo que Capital.
        É denotado pela letra $P$.
    \end{definition}
    \begin{definition}[Taxa de juro]
        A taxa de juro é um número que define quanto o capital aplicado rende em cada período.
        É denotado pela letra $i$.
    \end{definition}
%    \begin{definition}[Aplicação]
%        Aplicação é o nome dado à possibilidade de compra de um bem (ou papéis).
%        Dividem-se em renda fixa ou renda variável, sendo renda fixa quando a taxa de juro é fixa, e sendo 
%        renda variável caso contrário.
%        Ainda, dentro da renda variável
%    \end{definition}
    Para o cálculo de juro simples com uma iteração de tempo usamos a fórmula:
    $$J = P \cdot i$$
    \begin{example}
        Aplicar um capital de $R\$ \num{1000} $ com uma taxa de juros de $1\%$ ao mês (ou a.m.) por $1$ mês.
    \end{example}
    \begin{example}
        Aplicar um capital de $R\$ \num{1000} $ com uma taxa de juros de $10\%$ ao ano (ou a.a.) por $1$ ano.
    \end{example}
    \begin{remark}
        O cálculo só deverá ser efetuado $1$ vez. E é isto o que se quer dizer com "uma iteração de tempo".
    \end{remark}
    
    Para calcular quanto de juros é gerado por $n$ iterações 
    de tempo modificamos a fórmula e ficamos com:
    $$J = P \cdot i \cdot n$$
    Uma característica dos juros simples é que cada iteração o valor gerado de juros é o mesmo. 
    Consideremos o exemplo a seguir.

    \begin{example}
        Aplicar um capital de $R\$ \num{1000}$ com uma taxa de juros de $10\%$aa por $5$ anos.
        \begin{align*}
            \text{No primeiro ano temos}\ J_1 &= P \cdot i \cdot n = R\$ \num{1000} \cdot \num{.10} = R\$ 100 \\
            \text{No segundo ano temos}\ J_2 &= P \cdot i \cdot n = R\$ \num{1000} \cdot \num{.10} = R\$ 100 \\
            \text{No terceiro ano temos}\ J_3 &= P \cdot i \cdot n = R\$ \num{1000} \cdot \num{.10} = R\$ 100 \\
            \text{No quarto ano temos}\ J_4 &= P \cdot i \cdot n = R\$ \num{1000} \cdot \num{.10} = R\$ 100 \\
            \text{No quinto ano temos}\ J_5 &= P \cdot i \cdot n = R\$ \num{1000} \cdot \num{.10} = R\$ 100
        \end{align*}
        Onde $J_k$ representa o juro gerado na iteração $k$ (neste caso em cada ano).
    \end{example}

    Após a aplicação do capital, no ato da retirada não retiramos
    somente o juros gerado, mas também o principal aplicado.
    \begin{example}
        Ao aplicar $R\$ 500$ com taxa de $1\%$ am, por $12$ meses temos $J = R\$ 500 \cdot 0,01 \cdot 12 = R\$ 60 $. \\
        Complicado ao final do período poder retirar somente $R\$ 60$, não? Mas pode ser 
        retirado também o valor aplicado, neste caso $R\$ 500$.
    \end{example}
    \begin{definition}[Montante]
        É o valor que corresponde à soma do juro gerado com o capital aplicado.
        É denotado pela letra $M$.
    \end{definition}
        Naturalmente, é calculado pela fórmula: $M = J + P$.
    \begin{definition}[Vencimento]
        É a data mínima que o montante poderá ser retirado.
    \end{definition}
    Isso quer dizer que se uma aplicação tem vencimento em $1$ ano, ele não poderá \footnote{Varia de acordo 
    com a aplicação escolhida. Algumas têm período de carência, outras deixam retirar 
    parte do valor que seria obtido no vencimento. Alguns títulos podem ser vendidos no mercado secundário.
    } ser retirado antes disso.
        
    \section{Exercícios}
    Os exercícios de juro simples são mais ilustrativos do que ilustrativos!
    \begin{Exercise}[origin={exercicios.brasilescola.uol.com.br/}]
        Uma pessoa aplicou o capital de $R\$ \num{1200.00}$ a uma taxa de $2\%$ ao mês durante $14$ meses. 
        Determine os juros e o montante dessa aplicação.
    \end{Exercise}
    \begin{Exercise}[origin={exercicios.brasilescola.uol.com.br/}]
        Um capital aplicado a juros simples durante $2$ anos, sob taxa de juros de $5\%$ ao mês, 
        gerou um montante de $R\$ \num{26950.00}$. Determine o valor do capital aplicado. 
    \end{Exercise}

    %%%%%%%%%%%%%%%%%%%%%%%%%%%%%%%%%%%
    \chapter{Juro composto}
    No juro simples cada iteração gera o mesmo valor que todas as outras na mesma aplicação,
    porque cada iteração é calculada sobre o principal e sobre a taxa \footnote{Que são constantes, fixos}. 
    No juro composto isso não acontece, porque cada iteração é calculada sobre o montante da iteração anterior 
    e sobre a taxa
    \footnote{exeto na primeiríssima vez que é calculado imediatamente com o principal}.
    \begin{example}
        Vejamos uma tabela com o juro gerado por uma aplicação de $R\$ \num{1000}$ com uma taxa de $10\%am$ por $4$ meses.
        O tempo/iteração deve ser lido como "ao final da iteração 1" ou "ao final do mês 1" tem-se o juro $J_1$ e 
        o montante $M_1$.
        \begin{center}
           \begin{tabular}{|c|c|c|c|c|}
               \hline
               iteração (em meses) & i             & P (em R\$)       & J (em R\$)       & M (em R\$) \\
               \hline
               1        & \num{0.10}    & \num{1000}    & \num{100.00}     & \num{1100.00} \\
               2        & \num{0.10}    & \num{1000}    & \num{110.00}     & \num{1210.00} \\
               3        & \num{0.10}    & \num{1000}    & \num{121.00}     & \num{1331.00} \\
               4        & \num{0.10}    & \num{1000}    & \num{133.10}     & \num{1461.10} \\
               \hline
           \end{tabular}
        \end{center}
        Como é possível notar, cada nova iteração gera um juro levemente superior à iteração imediatamente anterior.
        Desta forma, calculamos o juro composto assim:
        \begin{align*}
            M_1 &= P   + J_1    &= P + P \cdot i       &                   &= P(1+i)                &= P(1+i)^1 \\
            M_2 &= M_1 + J_2    &= M_1 + M_1 \cdot i   &= M_1 (1 + i)      &= P(1+i)(1+i)           &= P(1+i)^2 \\
            M_3 &= M_2 + J_3    &= M_2 + M_2 \cdot i   &= M_2 (1 + i)      &= P(1+i)(1+i)(1+i)      &= P(1+i)^3 \\
            M_4 &= M_3 + J_4    &= M_3 + M_3 \cdot i   &= M_3 (1 + i)      &= P(1+i)(1+i)(1+i)(1+i) &= P(1+i)^4 \\
        \end{align*}
    \end{example}
        De maneira geral temos: 
        $$ M_n = M_{n-1} + J_n = M_{n-1} + M_{n-1} \cdot i = M_{n-1}(1 + i) = P(1+i)^n $$
        Portanto\footnote{Normalmente fica subentendido que é no vencimento, senão um período deve ser indicado.}, 
        para calcular quanto é o montante de uma aplicação usamos a fórmula: 
        $$M = P (1+i)^n$$
        
        Observe que usando a fórmula obtida, conseguimos calcular
        o montante num mês $n$ qualquer, sem ter que calcular os mêses anteriores,
        mas a fórmula que obtivemos é somente uma "abreviação" do "conceito de juro composto",
        e é o "conceito de juro composto" que utiliza-se do montante anterior. 

        \begin{example}
            Identificar o tempo mínimo que um capital de $R\$ \num{10000}$ deve ficar aplicado numa taxa
            de $1\%am$ para que o montante seja $R\$ \num{20000}$. 
            \begin{align}
                \num{20000} &= \num{10000}(1+i)^n \\
                2 \cdot \num{10000} &= \num{10000}(1+i)^n \\
                2 &= (1+i)^n  \\
                \log_{10} 2 &= \log_{10}(1+i)^n \\
                \log_{10} 2 &= n \cdot \log_{10}(1+i) \\
                \dfrac{\log_{10} 2}{\log_{10}(1+i)} &= n \\
                \log_{1+i} 2 &= n \\
                \log_{\num{1.01}} 2 &= n \approx \num{69.66}
            \end{align}
            No regime de juro simples com essa taxa precisaríamos de $100$ meses para duplicar a mesma quantia.
            No regime de juro composto precisamos somente de 69 meses e meio.
        \end{example}
    
        \section{Exercícios}

        \begin{Exercise}[origin={ENEM 2019, adaptado}]
            Uma pessoa fez um depósito inicial de $R\$ \num{200.00}$ em um fundo de Investimentos que possui 
            rendimento constante sob juros compostos de 5\% ao mês. Esse Fundo possui cinco planos de carência 
            (tempo mínimo necessário de rendimento do Fundo sem movimentação do cliente). Os planos são: \\
            • Plano A: carência de 10 meses;  \\
            • Plano B: carência de 15 meses;  \\
            • Plano C: carência de 20 meses;  \\
            • Plano D: carência de 28 meses;  \\
            • Plano E: carência de 40 meses.  \\
            O objetivo dessa pessoa é deixar essa aplicação rendendo até que o valor inicialmente aplicado duplique, 
            quando somado aos juros do fundo. Considere as aproximações: $\log 2 = 0,30$ e $\log 1,05 = 0,02$.
            Para que essa pessoa atinja seu objetivo apenas no período de carência, mas com a menor 
                carência possível, deverá optar pelo plano?
        \end{Exercise}
        \begin{Exercise}[origin={ENEM 2019, adaptado}]
            Uma pessoa se interessou em adquirir um produto anunciado em uma loja. Negociou com o gerente e
            conseguiu comprá-lo a uma taxa de juros compostos de $1\%$ ao mês. O primeiro pagamento será 
            um mês após a aquisição do produto, e no valor de $R\$ \num{202.00}$. O segundo pagamento será 
            efetuado um mês após o primeiro, e terá o valor de $R\$ \num{204.02}$. Para concretizar a 
            compra, o gerente emitirá uma nota fiscal com o valor do produto à vista negociado com o cliente,
            correspondendo ao financiamento aprovado.

            O valor à vista, em real, que deverá constar na nota fiscal é de
            \begin{enumerate}[A)]
                \item \num{398.02}.
                \item \num{400.00}.
                \item \num{401.94}.
                \item \num{404.00}.
                \item \num{406.02}.
            \end{enumerate}
        \end{Exercise}

        \begin{Exercise}[origin={ENEM 2019, adaptado}]
            Um contrato de empréstimo prevê que quando uma parcela é paga de forma antecipada, conceder-se-á uma redução 
            de juros de acordo com o período de antecipação. Nesse caso, paga-se o valor presente, que é o valor, naquele 
            momento, de uma quantia que deveria ser paga em uma data futura. Um valor presente $P$ submetido a juros compostos 
            com taxa $i$, por um período de tempo $n$, produz um valor futuro $V$ determinado pela fórmula
            $$V = P(1+i)^n$$
            Em um contrato de empréstimo com sessenta parcelas fixas mensais, de $R\$ \num{820.00}$, a uma taxa de juros
            de $\num{1.32}\%$ ao mês, junto com a trigésima parcela será paga antecipadamente uma outra parcela, 
            desde que o desconto seja superior a $25\%$ do valor da parcela.

            Utilize $0,2877$ como aproximação para $\ln \dfrac{4}{3}$ e $0,0131$ como aproximação 
            para $\ln 1,0132$. 
            A primeira das parcelas que poderá ser antecipada junto com a 30ª é a

            \begin{enumerate}[A)]
                \item 56ª.
                \item 55ª.
                \item 52ª.
                \item 51ª.
                \item 45ª.
            \end{enumerate}
        \end{Exercise}
    \chapter{Taxas de juros}
    As taxas tem seu período indicado, algumas são ao mês, ao bimestre, e assim por diante. 
    A comparação de taxas ou "transformação de taxas" baseia-se na ideia que em um período
    de tempo T, um mesmo capital deve gerar o mesmo juro.
    \begin{definition}{Taxas equivalentes}
        Taxas equivalentes são taxas que aplicadas sobre o mesmo capital, num mesmo 
        período de tempo, geram o mesmo juro.
    \end{definition}
    Para estes fins, denotarei por $i$ a taxa de juros de período menor e $I$ a de período maior.
    \begin{example}
        A juro simples, considere um capital de $R\$ \num{5000.00}$. Qual taxa anual é equivalente à taxa mensal de $1\%$? \\
        \begin{align}
            P i n = J = P I N \\
            \num{5000.00} \cdot 0,01 \cdot 12 = \num{5000.00} \cdot I \cdot 1 \\
            \num{600.00} = \num{5000.00} \cdot I  \\
            \dfrac{\num{600.00}}{\num{5000.00}} = I \\
            0,12 = I 
        \end{align}
    \end{example}
    Agora vamos verificar que a taxa equivalente pode ser encontrada sem conhecermos o juro, capital ou montante.
    Isto é, as taxas equivalentes se relacionam somente pelo seus valores e pelo tempo.
    \begin{align}
        P i n = J = P I N \\
        P i n = P I N \\
        i n = I N
    \end{align}
    Onde $n$ é o \emph{número de iterações} da taxa de "menor" tempo e $N$ o de "maior" tempo.
    \begin{example}
        Encontrar a taxa equivalente anual em juro simples de uma taxa de $1\%am$. \\
        \begin{align}
            i n = I N \\
            0,01 \cdot 12 = I \cdot 1 \\
            0,12 = I
        \end{align}
    \end{example}
    Para juro composto a ideia é a mesma: taxas equivalentes geram juro equivalente no mesmo tempo. Isso se expressa,
    no juro composto, como:
    \begin{align}
        P \cdot (1+I)^N &= P \cdot (1+i)^n \\
        (1+I)^N         &= (1+i)^n \label{eq-taxa-equivalentes-jc}
    \end{align}
    Repare que tem-se quatro incógnitas na fórmula \ref{eq-taxa-equivalentes-jc}, a saber $I$, $N$, $i$ e $n$. I 
    \begin{example}
        Encontrar a taxa equivalente anual em juro composto de uma taxa de $1\%am$. \\
        \begin{align}
            (1+I)^N = (1,01)^{n} \\
            (1+I)^1 = (1,01)^{12} \\
            1+I = 1,1268 \\
            I = 12,66\%aa 
        \end{align}
    \end{example}
    \begin{example}
        Encontrar a taxa mensal equivalente à taxa de $20\%aa$.
            (1+I)^N = (1,01)^{n} \\
    \end{example}

    \appendix
    \section{Números e funções Reais}
    O logaritmo é definido da seguinte forma: \\
    $$\log_a x = y \iff a^y = x$$
    quando $a, x, y$ são positivos (estritamente maiores que zero). O símbolo $\iff$ significa 
    que são equivalentes, é "semelhante" à uma igualdade, mas \\
    $5 + 10 + x = x + 15 \iff x+1 = x+1$, mas note que $15+x \neq x+1$.
    \begin{example}
        $\log_{10} 1000 = 3$, pois $10^3 = 1000$
    \end{example}
    A propriedade a seguir é útil para encontrar o tempo necessário para uma aplicação chegar em determinado valor.
    $$\log a^n = n \log a$$

\end{document}
