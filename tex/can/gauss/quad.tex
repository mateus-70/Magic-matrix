\documentclass{beamer}
 
\usepackage[utf8]{inputenc}
\usepackage{amsmath}
%\usepackage{fontspec}
%\setmainfont{Linux Libertine O}
%\fontfamily{Computer Modern}\selectfont{Computer Modern Serif}
%\newfontfamily\cmss
 
\usetheme{CambridgeUS} 
%Information to be included in the title page:
\title{Integração numérica: Quadraturas gaussianas}
\author[Mateus]
{Mateus Schroeder da Silva}

\institute{UDESC}
\date{06-2019}
 
%\logo{\includegraphics[height=1cm]{images/marca-udesc.png}}
\begin{document}
 
\frame{\titlepage}
%\begin{itemize}
    %\item $P_0(x) = 1$
    %\item $P_1(x) = x$
    %\item $P_2(x) = \frac{1}{2}\left( 3x^2-1 \right)$
    %\item $P_3(x) = \frac{1}{2}\left( 5x^3-3x \right)$
    %\item $P_4(x) = \frac{1}{8}\left( 35x^4-30x^2+3 \right)$
    %\item $P_5(x) = \frac{1}{8}\left( 63x^5-70x^3+15x \right)$
    %\item $P_6(x) = \frac{1}{16}\left( 231x^6-315x^4+105x^2 - 5 \right)$
%\end{itemize}
 
\begin{frame}
\frametitle{Integração numérica: Quadraturas Gaussianas}
The idea of Gaussian quadratures is to give ourselves the freedom to
choose not only the weighting coefficients, but also the location of the abscissas at
which the function is to be evaluated. They will no longer be equally spaced. Thus,
we will have twice the number of degrees of freedom at our disposal; it will turn out
that we can achieve Gaussian quadrature formulas whose order is, essentially, twice
    that of the Newton-Cotes formula with the same number of function evaluations. (PRESS et al, ANO)
\end{frame}

\begin{frame}
Does this sound too good to be true? Well, in a sense it is. The catch is a familiar
one, which cannot be overemphasized: High order is not the same as high accuracy.
High order translates to high accuracy only when the integrand is very smooth, in the
sense of being “well-approximated by a polynomial.”
\end{frame}


\begin{frame}
\frametitle{Dois pontos}
    Considerando a aproximação pelo método do trapézio $$\int_a^b {f(x)dx} \approx c_1f(a) + c_2f(b)$$ faremos uma extensão 
    a fim de obter a quadratura gaussiana de dois pontos.
    Suponhamos que o argumento das funções não seja pré-determinado, temos $$\int_a^b {f(x)dx} \approx c_1f(x_1) + c_2f(x_2)$$
\end{frame}


\begin{frame}
\frametitle{}
    São 4 as incógnitas: $c_1, c_2, x_1, x_2$ que são encontradas supondo que o método integre uma 
    função de terceiro grau $f(x) = a_0 + a_1x + a_2x^2 + a_3x^3$ com erro zero. Então:
\end{frame}

\begin{frame}
\frametitle{}
    $$\int_a^b {f(x)dx} = \int_a^b {\left( a_0 + a_1x + a_2x^2 + a_3x^3 \right)dx} = $$
    $$ \left[ a_0x + a_1 \dfrac{x^2}{2} +  a_2 \dfrac{x^3}{3} +  a_3 \dfrac{x^4}{4} \right]_a^b = $$ 
    $$a_0 (b-a) + a_1 \left( \dfrac{b^2 - a^2}{2}\right) + a_2 \left( \dfrac{b^3 - a^3}{3}\right) + a_3 \left( \dfrac{b^4 - a^4}{4}\right) = $$
    $$ c_1 \left( a_0 + a_1x_1 + a_2x_1^2 + a_3x_1^3 \right) + c_2 \left( a_0 + a_1x_2 + a_2x_2^2 + a_3x_2^3 \right) $$
    $$ a_0 \left( c_1 + c_2 \right) +  a_1 \left( c_1x_1 + c_2x_2 \right) + a_2 \left( c_1x_1^2 + c_2x_2^2 \right) +  a_3 \left( c_1x_1^3 + c_2x_2^3 \right) $$
\end{frame}

\begin{frame}
\frametitle{}
Obtemos então o sistema não-linear
    $$ b - a = c_1 + c_2$$
    $$ \dfrac{b^2 - a^2}{2} = c_1x_1 + c_2x_2$$
    $$ \dfrac{b^3 - a^3}{3} = c_1x_1^2 + c_2x_2^2$$
    $$ \dfrac{b^4 - a^4}{4} = c_1x_1^3 + c_2x_2^3$$
\end{frame}

\begin{frame}
\frametitle{}
    Tal sistema tem somente uma solução, a saber:
    $$ c_1 = \frac{b-a}{2} $$
    $$ c_2 = \frac{b-a}{2} $$
    $$ x_1 = \left( \frac{b-a}{2} \right) \left( -\frac{1}{\sqrt{3}}\right) + \frac{b-a}{2} $$
    $$ x_1 = \left( \frac{b-a}{2} \right) \left( \frac{1}{\sqrt{3}}\right) + \frac{b-a}{2} $$
\end{frame}

\begin{frame}
\frametitle{Mas e aí?}
Esse método aproxima bem em todos os casos?
\end{frame}
\begin{frame}
\frametitle{Mas e aí?}
Não. 
\end{frame}

\begin{frame}
\frametitle{Mas e aí?}
O caso mostrado funciona bem quando a função que queremos integrar no intervalo limitado, fechado pode ser aproximada
com satisfatoriedade por uma função polinomial. O caso é semelhate à quadratura Gauss-Legendre, 
que utiliza como os pesos, os $"c_n"$ valores obtidos com os polinômios de Legendre.
\end{frame}

\begin{frame}
\frametitle{Mas e aí?}
Gauss-Jacobi
    $$ f(x) = (1 - x)^\alpha (1+x)^\beta g(x), \alpha, \beta > -1$$
    Utilizada para calcular funções com pontos singulares (locais estranhos em algumas funções, por exemplo pontos de descontinuidade ou de não-derivabilidade)
\end{frame}
\begin{frame}
\frametitle{Mas e aí?}
Chebishev-Gauss
    Usada para calcular funções da forma:
    $$\int_{-1}^1 {\dfrac{f(x)}{\sqrt{1-x^2}} dx}$$
    $$\int_{-1}^1 {\sqrt{1-x^2} g(x) } dx$$
\end{frame}
%\begin{frame}
%\frametitle{Integração numérica: Quadraturas Gaussianas}
    %$$\int_a^b{f(x) dx} \approx A_0f(x_0) + \cdots + A_nf(x_n)$$
    %onde $x_0, x_1, \cdots, x_n$ são $n + 1$ pontos distintos. Tais fórmulas são exatas
    %para polinômios de grau $\leq 2n + 1$ e são conhecidas como quadratura Gaussiana.
%\end{frame}


%\frame{\titlepage}
%\begin{frame}
%\frametitle{}
%Em matemática, em especial na análise numérica, existe uma grande família de algoritmos, cujo principal objetivo é aproximar o valor de uma dada integral definida de uma função sem o uso de uma expressão analítica para a sua primitiva.[1]
%
%Normalmente, estes métodos adotam as seguintes três fases:[2]
%
%Decomposição do domínio em pedaços (um intervalo contido de sub-intervalos);\\
%Integração aproximada da função de cada pedaço;\\
%Soma dos resultados numéricos obtidos. \\
%\end{frame}
%
%\frame{\titlepage}
%\begin{frame}
%A necessidade de se usar a integração numérica surge de razões como:[2][3]
%
%nem todas as funções admitem uma primitiva de forma explícita (por exemplo, a função erro);
%a primitiva da função é muito complicada para ser avaliada;
%quando não se dipões de uma expressão analítica para o integrando, mas se conhece seus valores em um conjunto de pontos do domínio.
%O método básico envolvido nesta aproximação é chamado de quadratura numérica e consiste na seguinte expressão:
%
%$${\displaystyle \int _{a}^{b}f(x)dx\simeq \sum _{i=0}^{n}\alpha _{i}f(x_{i})\,} \int _{a}^{b}f(x)dx\simeq \sum _{{i=0}}^{{n}}\alpha _{i}f(x_{i})$$,
%onde ${\displaystyle \{\alpha _{i}\}\,} \{\alpha _{i}\}$, são coeficientes reais (chamados de pesos da quadratura) e ${\displaystyle \{x_{i}\}\,} \{x_{i}\}$, são pontos de $\displaystyle [a,b]$, $\displaystyle [a,b]$,(chamados de pontos da quadratura) .[2]
%\end{frame}
%
					
%\begin{frame}
%\frametitle{}
%\end{frame}
					
%\begin{frame}
%\frametitle{}
%\end{frame}
					
%\begin{frame}
%\frametitle{}
%\end{frame}
					
%\begin{frame}
%\frametitle{}
%\end{frame}
					
%\begin{frame}
%\frametitle{}
%\end{frame}
					
%\begin{frame}
%\frametitle{}
%\end{frame}
					
%\begin{frame}
%\frametitle{}
%\end{frame}
					
%\begin{frame}
%\frametitle{}
%\end{frame}
					
%\begin{frame}
%\frametitle{}
%\end{frame}
					
%\begin{frame}
%\frametitle{}
%\end{frame}
\begin{frame}
\frametitle{Referências}
NUMERICAL RECIPES: The art of science computing. Press et al. Cambridge, 2007\\
<http://mathforcollege.com/nm/mws/gen/07int/mws\_gen\_int\_txt\_gaussquadrature.pdf> Acesso em 12-06-2019.\\
CÁLCULO NUMÉRICO: Aspectos teóricos e computacionais. Ruggiero e Lopes. Pearson, 2ed, São Paulo.
\end{frame}
%\begin{frame}
%\frametitle{Derivação da fórmula de área}
%\end{frame}
%\begin{frame}
%\frametitle{Derivação da fórmula de área}
%\end{frame}
\end{document}
