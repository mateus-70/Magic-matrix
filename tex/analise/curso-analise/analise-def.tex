\documentclass{article}
\usepackage{amsmath}
\usepackage{amssymb}
\usepackage{amsthm}
%\usepackage[portuges]{babel}
\usepackage[utf8]{inputenc}
\usepackage[T1]{fontenc}

%\let\oldln\ln
\renewcommand{\ln}[1]{ \oldln {(#1)}}
\newcommand{\abs}[1]{\lvert #1 \rvert}

%\let\oldint\int
%\newcommand{\fint}[3][s][s]{ \int_{#1}^{#2}{\left(#3\right)}}

%\newcommand{\dint}[4]{ \int_{#1}^{#2} {\left(#3\right)}\, \mathrm{d}#4}

\newcommand{\paren}[1]{ \left( #1 \right)}
\newcommand{\colch}[1]{ \left[ #1 \right]}
\newcommand{\tr}[1]{\mathrm{tr} \left( #1 \right)}
\let\olddet\det
\renewcommand{\det}[1]{\olddet{ \left( #1 \right)} }

%\providecommand{\sen}{\mathrm{--ERRO--}}
\providecommand{\senh}{\mathrm{--ERRO--}}
\providecommand{\arcsen}{\mathrm{--ERRO--}}
\providecommand{\arsenh}{\mathrm{--ERRO--}}

\renewcommand{\sen}[1]{\mathrm{sen}\, \left( #1 \right)}
\renewcommand{\arcsen}[1]{\mathrm{arcsen}\, \left( #1 \right)}
\renewcommand{\senh}[1]{\mathrm{senh}\, \left( #1 \right)}
\renewcommand{\arsenh}[1]{\mathrm{arsenh}\, \left( #1 \right)}

\providecommand{\cos}{\mathrm{--ERRO--}}
\providecommand{\cosh}{\mathrm{--ERRO--}}
\providecommand{\arccos}{\mathrm{--ERRO--}}
\providecommand{\arcosh}{\mathrm{--ERRO--}}

\renewcommand{\cos}[1]{\mathrm{cos}\, \left( #1 \right)}
\renewcommand{\cosh}[1]{\mathrm{cosh}\, \left( #1 \right)}
\renewcommand{\arccos}[1]{\mathrm{arccos}\, \left( #1 \right)}
\renewcommand{\arcosh}[1]{\mathrm{arcosh}\, \left( #1 \right)}

\providecommand{\tg}{\mathrm{--ERRO--}}
\providecommand{\tgh}{\mathrm{--ERRO--}}
\providecommand{\arctg}{\mathrm{--ERRO--}}
\providecommand{\artgh}{\mathrm{--ERRO--}}

\renewcommand{\tg}[1]{\mathrm{tg}\,\left( #1 \right)}
\renewcommand{\tgh}[1]{\mathrm{tgh}\, \left( #1 \right)}
\renewcommand{\arctg}[1]{\mathrm{arctg}\, \left( #1 \right)}
\renewcommand{\artgh}[1]{\mathrm{artg}\, \left( #1 \right)}

\providecommand{\sec}{\mathrm{--ERRO--}}
\providecommand{\cossec}{\mathrm{--ERRO--}}
\providecommand{\cotg}{\mathrm{--ERRO--}}

\renewcommand{\sec}[1]{\mathrm{sec}\, \left( #1 \right)}
\renewcommand{\cossec}[1]{\mathrm{cossec}\, \left( #1 \right)}
\renewcommand{\cotg}[1]{\mathrm{cotg}\, \left( #1 \right)}

\providecommand{\arcsec}{\mathrm{--ERRO--}}
\providecommand{\arccossec}{\mathrm{--ERRO--}}
\providecommand{\arccotg}{\mathrm{--ERRO--}}

\renewcommand{\arcsec}[1]{\mathrm{sec}\, \left( #1 \right)}
\renewcommand{\arccossec}[1]{\mathrm{cossec}\, \left( #1 \right)}
\renewcommand{\arccotg}[1]{\mathrm{cotg}\, \left( #1 \right)}

\providecommand{\sech}{\mathrm{--ERRO--}}
\providecommand{\cossech}{\mathrm{--ERRO--}}
\providecommand{\cotgh}{\mathrm{--ERRO--}}

\renewcommand{\sech}[1]{\mathrm{sech}\, \left( #1 \right)}
\renewcommand{\cossech}[1]{\mathrm{cossech}\, \left( #1 \right)}
\renewcommand{\cotgh}[1]{\mathrm{cotgh}\, \left( #1 \right)}

\providecommand{\arsech}{\mathrm{--ERRO--}}
\providecommand{\arcossech}{\mathrm{--ERRO--}}
\providecommand{\arcotgh}{\mathrm{--ERRO--}}

\renewcommand{\arsech}[1]{\mathrm{arsech}\, \left( #1 \right)}
\renewcommand{\arcossech}[1]{\mathrm{arcossech}\, \left( #1 \right)}
\renewcommand{\arcotgh}[1]{\mathrm{arcotgh}\, \left( #1 \right)}


% plain definition remark em theoremstyle
    \theoremstyle{definition}
    \newtheorem{definition}{Definição}
    
    \theoremstyle{remark}
    \newtheorem*{remark}{Nota}

    \theoremstyle{plain}
    \newtheorem{theorem}{Teorema}

    \theoremstyle{plain}
    \newtheorem{corolary}{Corolário}

    \theoremstyle{plain}
    \newtheorem{lemma}{Lema}
\begin{document}

    \theoremstyle{definition}
    \begin{definition}{Ponto interior}
        Dado um conjunto  $X \subset\mathbb{R}$, um ponto $x \in X$\ chama-se ponto interior
        de $X$ quando existe um intervalo aberto $(a,b)$ tal que $x \in (a,b) \subset X$
   \end{definition}

    \begin{definition}{Conjunto aberto}
       Diz-se que $X$ é um conjunto aberto, se $int X = X$ 
    \end{definition}

    \begin{definition}{Ponto aderente}
        Diz-se que um ponto $a$ é aderente a $X \subset \mathbb{R}$
        quando $a$ for limite de uma sequência $x_n \in X$.
    \end{definition}
    \begin{remark}
        Todo ponto de $X$ é aderente a $X$.
    \end{remark}
    \begin{definition}{Fecho}
        Chamaremos de fecho do conjunto $X$ ao conjunto $XXX$ formado pelos pontos aderentes a $X$.
    \end{definition}
    \begin{definition}{Conjunto fechado}
        É quando o conjunto é igual ao seu fecho.
    \end{definition}
    \begin{definition}{Denso}
        Sejam $X$,$Y$ conjuntos de números reais, com $X \subset Y$. 
        Diremos que $X$ é denso em $Y$ quando todo ponto de $Y$ for aderente a $X$.
    \end{definition}
    \begin{definition}{Ponto de acumulação}
        Seja $X \subset \mathbb{R}$. Um número $a \in \mathbb{X}$ chama-se 
        ponto de acumulação do conjunto $X$ quando todo intervalo aberto
        (a-\epsilon, a+\epsilon), de centro $a$, contém algum ponto $x \in X$ diferente de $a$.
        O conjunto dos pontos de acumulação de $X$ será representado pela notação $X'$ (e, às vezes,
        chamado o derivado de $X$).
        A condição $a \in X'$ ($a$ é ponto de acumulação de $X$) exprime-se simboliamente do modo seguinte: \\
        \forall\epsilon > 0 \exists x \in X; 0 < \abs{x - a} < \epsilon
    \end{definition}
    \begin{definition}{Ponto isolado}
        Um ponto $a \in X$ que não é ponto de acumulação de $X$ chama-se um ponto isolado de $X$.
    \end{definition}
    \begin{definition}{Ponto de acumulação à direita}
        Dizemos que $a$ é ponto de acumulação à direita do conjunto $X$ quando todo intervalo $[a, a+\epsilon)$,
        com $\epsilon > 0$, contém algum ponto de $X$ diferente de $a$.
        Equivalentemente, todo intervalo $[a, a+\epsilon)$ contém algum ponto de $X$ diferente de $a$, ou então 
            que $a$ é ponto de acumulação de $X \cap [a, +\infty)$.
        Outra afirmação equivalente a esta é dizer que $a$ é limite de uma sequência decrescente de pontos de $X$.
        Notação: ${X'}_+$
    \end{definition}

    \newpage
    \begin{definition}{Cobertura}
        Uma cobertura de um conjunto $X \subset \mathbb{R}$ é uma família $C = \left( C_\lambda \right)_{\lambda \in L}$
        de conjuntos $C_\lambda \in \mathbb{R}$, tais que $X \subset \bigcup\limits_{\lambda \in L} C_\lambda$,
        isto é, para todo 
    \end{definition}

    ----
    
    \begin{definition}{}
    \end{definition}
\end{document}
