\documentclass{article}
\usepackage{amsmath}
\usepackage{amssymb}
\usepackage{amsthm}
%\usepackage[portuges]{babel}
\usepackage[utf8]{inputenc}
\usepackage[T1]{fontenc}

%\let\oldln\ln
\renewcommand{\ln}[1]{ \oldln {(#1)}}
\newcommand{\abs}[1]{\lvert #1 \rvert}

%\let\oldint\int
%\newcommand{\fint}[3][s][s]{ \int_{#1}^{#2}{\left(#3\right)}}

%\newcommand{\dint}[4]{ \int_{#1}^{#2} {\left(#3\right)}\, \mathrm{d}#4}

\newcommand{\paren}[1]{ \left( #1 \right)}
\newcommand{\colch}[1]{ \left[ #1 \right]}
\newcommand{\tr}[1]{\mathrm{tr} \left( #1 \right)}
\let\olddet\det
\renewcommand{\det}[1]{\olddet{ \left( #1 \right)} }

%\providecommand{\sen}{\mathrm{--ERRO--}}
\providecommand{\senh}{\mathrm{--ERRO--}}
\providecommand{\arcsen}{\mathrm{--ERRO--}}
\providecommand{\arsenh}{\mathrm{--ERRO--}}

\renewcommand{\sen}[1]{\mathrm{sen}\, \left( #1 \right)}
\renewcommand{\arcsen}[1]{\mathrm{arcsen}\, \left( #1 \right)}
\renewcommand{\senh}[1]{\mathrm{senh}\, \left( #1 \right)}
\renewcommand{\arsenh}[1]{\mathrm{arsenh}\, \left( #1 \right)}

\providecommand{\cos}{\mathrm{--ERRO--}}
\providecommand{\cosh}{\mathrm{--ERRO--}}
\providecommand{\arccos}{\mathrm{--ERRO--}}
\providecommand{\arcosh}{\mathrm{--ERRO--}}

\renewcommand{\cos}[1]{\mathrm{cos}\, \left( #1 \right)}
\renewcommand{\cosh}[1]{\mathrm{cosh}\, \left( #1 \right)}
\renewcommand{\arccos}[1]{\mathrm{arccos}\, \left( #1 \right)}
\renewcommand{\arcosh}[1]{\mathrm{arcosh}\, \left( #1 \right)}

\providecommand{\tg}{\mathrm{--ERRO--}}
\providecommand{\tgh}{\mathrm{--ERRO--}}
\providecommand{\arctg}{\mathrm{--ERRO--}}
\providecommand{\artgh}{\mathrm{--ERRO--}}

\renewcommand{\tg}[1]{\mathrm{tg}\,\left( #1 \right)}
\renewcommand{\tgh}[1]{\mathrm{tgh}\, \left( #1 \right)}
\renewcommand{\arctg}[1]{\mathrm{arctg}\, \left( #1 \right)}
\renewcommand{\artgh}[1]{\mathrm{artg}\, \left( #1 \right)}

\providecommand{\sec}{\mathrm{--ERRO--}}
\providecommand{\cossec}{\mathrm{--ERRO--}}
\providecommand{\cotg}{\mathrm{--ERRO--}}

\renewcommand{\sec}[1]{\mathrm{sec}\, \left( #1 \right)}
\renewcommand{\cossec}[1]{\mathrm{cossec}\, \left( #1 \right)}
\renewcommand{\cotg}[1]{\mathrm{cotg}\, \left( #1 \right)}

\providecommand{\arcsec}{\mathrm{--ERRO--}}
\providecommand{\arccossec}{\mathrm{--ERRO--}}
\providecommand{\arccotg}{\mathrm{--ERRO--}}

\renewcommand{\arcsec}[1]{\mathrm{sec}\, \left( #1 \right)}
\renewcommand{\arccossec}[1]{\mathrm{cossec}\, \left( #1 \right)}
\renewcommand{\arccotg}[1]{\mathrm{cotg}\, \left( #1 \right)}

\providecommand{\sech}{\mathrm{--ERRO--}}
\providecommand{\cossech}{\mathrm{--ERRO--}}
\providecommand{\cotgh}{\mathrm{--ERRO--}}

\renewcommand{\sech}[1]{\mathrm{sech}\, \left( #1 \right)}
\renewcommand{\cossech}[1]{\mathrm{cossech}\, \left( #1 \right)}
\renewcommand{\cotgh}[1]{\mathrm{cotgh}\, \left( #1 \right)}

\providecommand{\arsech}{\mathrm{--ERRO--}}
\providecommand{\arcossech}{\mathrm{--ERRO--}}
\providecommand{\arcotgh}{\mathrm{--ERRO--}}

\renewcommand{\arsech}[1]{\mathrm{arsech}\, \left( #1 \right)}
\renewcommand{\arcossech}[1]{\mathrm{arcossech}\, \left( #1 \right)}
\renewcommand{\arcotgh}[1]{\mathrm{arcotgh}\, \left( #1 \right)}


% plain definition remark em theoremstyle
    \theoremstyle{remark}
    \newtheorem*{remark}{Nota}

    \theoremstyle{plain}
    \newtheorem{theorem}{Teorema}

    \theoremstyle{plain}
    \newtheorem{corolary}{Corolário}

    \theoremstyle{plain}
    \newtheorem{lemma}{Lema}
\begin{document}
    \begin{theorem}
        \begin{enumerate}
        \item Se $\mathrm{A_1 \subset \mathbb{R} }$ e $ \mathrm{ A_2 \subset \mathbb{R} }$ são abertos, então $\mathrm{ A_1 \cup A_2 }$ é aberto.
        \item Seja $\mathrm{ \left( A_\lambda \right)_{\lambda \in L} }$ 
            uma família arbitrária de conjuntos abertos $\mathrm{ A_\lambda \subset \mathbb{R} }$.
            A reunião $\mathrm{ A = \bigcup\limits_{\lambda \in L} A_\lambda }$
        \end{enumerate}
    \end{theorem}
    \begin{corolary}
        A interseção finita de conjuntos abertos é um conjunto aberto.
    \end{corolary}
    \begin{theorem}[Estrutura dos abertos da reta]
        Todo subconjunto aberto $\mathrm{A \subset \mathbb{R} }$ se exprime, de modo único, 
        como uma reunião enumerável de intervalos abertos, dois a dois disjuntos. 
        \begin{lemma}
            Seja $\mathrm{ \left( I_\lambda \right)_{\lambda \in L} }$ uma família de intervalos abertos, 
            todos contendo o ponto $\mathrm{p \in \mathbb{R} }$. 
            Então $\mathrm{I = \bigcup\limits_{\lambda \in L} I_\lambda }$ é um intervalo aberto.
        \end{lemma}
    \end{theorem}
    \begin{corolary}
        Seja $\mathrm I$ um intervalo aberto. $\mathrm{ I = A \cup B }$, 
        onde $\mathrm A$ e $\mathrm B$ são conjuntos abertos disjuntos, então um desses conjuntos é igual a $\mathrm I$ e o outro é vazio.
    \end{corolary}
    \begin{theorem}
        Um ponto $a \in \mathbb{R}$ é aderente a um conjunto $X \subset \mathbb{R}$ se, e somente se, 
        para todo $\epsilon > 0 $ tem-se $X \cap ( a-\epsilon, a+ \epsilon ) \neq \not 0$.
    \end{theorem}
    \begin{corolary}
        Sejam $X \in \mathbb{R}$ limitado inferiormente e $Y \in \mathbb{R}$ limitado superiormente. 
        Então $a = inf X$ é aderente a $X$ e $b = sup Y$ é aderente a $Y$.
    \end{corolary}
    \begin{theorem}
        Um conjunto $F \in \mathbb{R}$ é fechado se, e somente se, seu complementar $R-F$ é aberto.
    \end{theorem}
    \begin{corolary}
        \begin{itemize}
            \item $\mathbb{R}$ é o conjunto vazio são fechados.
            \item Se $F_1, F_2, \dots, F_n$ são fechados, então $F_1 \cup F_2 \cup \dots \cup F_n$ é fechado.
            \item Se $ \left( F_\lambda \right)_{\lambda \in L}$ é uma família quaquer de conjuntos fechados então a interseção
                $F = \bigcap\limits_{\lambda \in L} F_\lambda$ é um conjunto fechado.
        \end{itemize}
    \end{corolary}
    \begin{theorem}
        O fecho de todo conjunto $X \in \mathbb{R}$ é um conjunto fechado, isto é, pg 172.
    \end{theorem}
    \begin{theorem}
        Todo conjunto $X$ de números reais contém um subconjunto enumerável $E$, denso em $X$.
    \end{theorem}
    \begin{theorem}
        Dados $X \subset \mathbb{R}$ e $a \in \mathbb{R}$, as seguintes afirmações são equivalentes:
        \begin{itemize}
            \item $a \in X'$;
            \item $a = lim x_n$, onde $(x_n)$ é uma sequência de elementos de $X$, dois a dois distintos;
            \item todo intervalo aberto contendo $a$ possui uma infinidade de elementos de $X$.
        \end{itemize}
    \end{theorem}
    \begin{corolary}
        Se $X' \neq vazio, pg 176$ então $X$ é infinito.
    \end{corolary}
    \begin{theorem}
        Para todo $X \subset \mathbb{R}$, tem-se pg 177, teo 8. 
        Ou seja, o fecho de um conjunto $X$ é obtido acrescentando-se a $X$ os seus pontos de acumulação.
    \end{theorem}
    \begin{corolary}
        X é fechado se, e somente se, $X' \in X$.
    \end{corolary}
    \begin{corolary}
        Se todos os pontos do conjunto $X$ são isolados então $X$ é enumerável.
    \end{corolary}
    \begin{theorem}
        Seja $F \subset \mathbb{R}$ não-vazio tal que $F = F'$. (Isto é, $F$ é um conjunto fechado não-vazio
        sem pontos isolados.) Então $F$ é não enumerável.
    \end{theorem}
    \begin{lemma}
        Seja $F$ fechado, não-vazio, sem pontos isolados. para todo $x \in \mathbb{R}$ 
        existe $F_x$ limitado, fechado, não-vazio, sem pontos isolados, tal que $x \not\in F_x \subset F$.
    \end{lemma}
    \begin{corolary}[Equivalente ao teorema]
        Todo conjunto fechado enumerável não-vazio possui algum ponto isolado.
    \end{corolary}
    \begin{corolary}
        O conjunto de Cantor é não-enumerável.
    \end{corolary}

    ----
    \begin{corolary}
    \end{corolary}
    
    \begin{theorem}
    \end{theorem}
    
\end{document}
