\documentclass{article}
\usepackage{amsmath}
\usepackage{amssymb}
\usepackage{amsthm}
%\usepackage[portuges]{babel}
\usepackage[utf8]{inputenc}
\usepackage[T1]{fontenc}
\usepackage{enumitem}


\let\oldln\ln
\renewcommand{\ln}[1]{ \oldln {(#1)}}
\newcommand{\abs}[1]{\lvert #1 \rvert}

%\let\oldint\int
%\newcommand{\fint}[3][s][s]{ \int_{#1}^{#2}{\left(#3\right)}}

%\newcommand{\dint}[4]{ \int_{#1}^{#2} {\left(#3\right)}\, \mathrm{d}#4}

\newcommand{\paren}[1]{ \left( #1 \right)}
\newcommand{\colch}[1]{ \left[ #1 \right]}
\newcommand{\tr}[1]{\mathrm{tr} \left( #1 \right)}
\let\olddet\det
\renewcommand{\det}[1]{\olddet{ \left( #1 \right)} }

\providecommand{\sen}{\mathrm{--ERRO--}}
\providecommand{\senh}{\mathrm{--ERRO--}}
\providecommand{\arcsen}{\mathrm{--ERRO--}}
\providecommand{\arsenh}{\mathrm{--ERRO--}}

\renewcommand{\sen}[1]{\mathrm{sen}\, \left( #1 \right)}
\renewcommand{\arcsen}[1]{\mathrm{arcsen}\, \left( #1 \right)}
\renewcommand{\senh}[1]{\mathrm{senh}\, \left( #1 \right)}
\renewcommand{\arsenh}[1]{\mathrm{arsenh}\, \left( #1 \right)}

\providecommand{\cos}{\mathrm{--ERRO--}}
\providecommand{\cosh}{\mathrm{--ERRO--}}
\providecommand{\arccos}{\mathrm{--ERRO--}}
\providecommand{\arcosh}{\mathrm{--ERRO--}}

\renewcommand{\cos}[1]{\mathrm{cos}\, \left( #1 \right)}
\renewcommand{\cosh}[1]{\mathrm{cosh}\, \left( #1 \right)}
\renewcommand{\arccos}[1]{\mathrm{arccos}\, \left( #1 \right)}
\renewcommand{\arcosh}[1]{\mathrm{arcosh}\, \left( #1 \right)}

\providecommand{\tg}{\mathrm{--ERRO--}}
\providecommand{\tgh}{\mathrm{--ERRO--}}
\providecommand{\arctg}{\mathrm{--ERRO--}}
\providecommand{\artgh}{\mathrm{--ERRO--}}

\renewcommand{\tg}[1]{\mathrm{tg}\,\left( #1 \right)}
\renewcommand{\tgh}[1]{\mathrm{tgh}\, \left( #1 \right)}
\renewcommand{\arctg}[1]{\mathrm{arctg}\, \left( #1 \right)}
\renewcommand{\artgh}[1]{\mathrm{artg}\, \left( #1 \right)}

\providecommand{\sec}{\mathrm{--ERRO--}}
\providecommand{\cossec}{\mathrm{--ERRO--}}
\providecommand{\cotg}{\mathrm{--ERRO--}}

\renewcommand{\sec}[1]{\mathrm{sec}\, \left( #1 \right)}
\renewcommand{\cossec}[1]{\mathrm{cossec}\, \left( #1 \right)}
\renewcommand{\cotg}[1]{\mathrm{cotg}\, \left( #1 \right)}

\providecommand{\arcsec}{\mathrm{--ERRO--}}
\providecommand{\arccossec}{\mathrm{--ERRO--}}
\providecommand{\arccotg}{\mathrm{--ERRO--}}

\renewcommand{\arcsec}[1]{\mathrm{sec}\, \left( #1 \right)}
\renewcommand{\arccossec}[1]{\mathrm{cossec}\, \left( #1 \right)}
\renewcommand{\arccotg}[1]{\mathrm{cotg}\, \left( #1 \right)}

\providecommand{\sech}{\mathrm{--ERRO--}}
\providecommand{\cossech}{\mathrm{--ERRO--}}
\providecommand{\cotgh}{\mathrm{--ERRO--}}

\renewcommand{\sech}[1]{\mathrm{sech}\, \left( #1 \right)}
\renewcommand{\cossech}[1]{\mathrm{cossech}\, \left( #1 \right)}
\renewcommand{\cotgh}[1]{\mathrm{cotgh}\, \left( #1 \right)}

\providecommand{\arsech}{\mathrm{--ERRO--}}
\providecommand{\arcossech}{\mathrm{--ERRO--}}
\providecommand{\arcotgh}{\mathrm{--ERRO--}}

\renewcommand{\arsech}[1]{\mathrm{arsech}\, \left( #1 \right)}
\renewcommand{\arcossech}[1]{\mathrm{arcossech}\, \left( #1 \right)}
\renewcommand{\arcotgh}[1]{\mathrm{arcotgh}\, \left( #1 \right)}


\begin{document}
\begin{center}
    Lista 1
\end{center}
Assunto: Função exponencial.\\
Professor Mateus Schroeder da Silva\\
Discente: \\

\begin{enumerate}
    \item
        Calcule se possível, senão justifique. Uso de propriedades é opcional. Podem conferir as respostas com calculadora. \\
        \begin{enumerate}[label=\alph*)]
            \item $2^2$
            \item $2^4$
            \item $(2 \cdot 3)^3$
            \item $(2+3)^2$
            \item $(2 \cdot 3)^2$
            \item $0^1$
            \item $3^0$
            \item $1^0$
            \item $0^{\frac{2}{3}}$
            \item $0^0$
          \end{enumerate}
    \item Calcule a expressão: $7^2 \cdot 7^3 \cdot 3 - 3^5 \cdot 49$. Dica: fatore os números e evidencie o que for comum.
    \item Escreva como produto de potências
        \begin{enumerate}[label=\alph*)]
            \item $5^{x+2}$ 
            \item $5^{x-2}$
            \item $5^{\frac{5}{3} - \frac{x}{2}}$
        \end{enumerate}
    \item Calcule o valore de $x$ para a equação:
        $3^{2*x} = 729$
    \item Considere as equações:
        \begin{equation}
            (-2)^2 = 2^2
        \end{equation}
        \begin{equation}
            -2 = 2
        \end{equation}
        \begin{enumerate}[label=\alph*)]
           \item A segunda equação é verdadeira? O que aconteceu quando concluimos a segunda equação a partir da primeira? Cometemos um erro? Qual?
           \item Seja $ |a| = |b|$, o que podemos concluir a respeito de
               $ a^y$ e $b^y $ se $y$ for um número par?
        \end{enumerate}
    \item Escreva como potências de 3.
        \begin{enumerate}[label=\alph*)]
            \item $3^{5+1}$
            \item ${(3^6)}^2$
            \item $3^{6^2}$
            \item $\frac{3}{9}$
            \item $\frac{27}{3^2}$
        \end{enumerate}
    \item (DANTE adaptado) Calcule o valor de: 
        \begin{equation*} 
            { \left( {\sqrt{2}}^{\sqrt{2} } \right) }^{\sqrt{2}}
        \end{equation*}
        \begin{equation*} 
            { {\sqrt{2}}^ {\left( {\sqrt{2} }^{\sqrt{2}} \right) }}
        \end{equation*}
    \item (DANTE adaptado) Determine o valor da seguinte expressão: $ 1^{-\pi} + 0^{\sqrt{5}}$
    \item Qual número é maior? $2^{300}$ ou $3^{200}$? Justifique.
    \item Simplifique
        $$ \dfrac{ {2^{\frac{2}{3}}} \cdot {3^{\frac{3}{2}}} \cdot {\left( {\frac{1}{2} } \right)}^{-4}}{\frac{1}{2^{\frac{3}{2}} \cdot 3^{-1}}} $$
\end{enumerate}
\end{document}
