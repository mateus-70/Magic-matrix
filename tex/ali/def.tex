\documentclass{article}
\usepackage{amsmath}
\usepackage{amssymb}
\usepackage{amsthm}
%\usepackage[portuges]{babel}
\usepackage[utf8]{inputenc}
\usepackage[T1]{fontenc}

\let\oldln\ln
\renewcommand{\ln}[1]{ \oldln {(#1)}}
\newcommand{\abs}[1]{\lvert #1 \rvert}

%\let\oldint\int
%\newcommand{\fint}[3][s][s]{ \int_{#1}^{#2}{\left(#3\right)}}

%\newcommand{\dint}[4]{ \int_{#1}^{#2} {\left(#3\right)}\, \mathrm{d}#4}

\newcommand{\paren}[1]{ \left( #1 \right)}
\newcommand{\colch}[1]{ \left[ #1 \right]}
\newcommand{\tr}[1]{\mathrm{tr} \left( #1 \right)}
\let\olddet\det
\renewcommand{\det}[1]{\olddet{ \left( #1 \right)} }

\providecommand{\sen}{\mathrm{--ERRO--}}
\providecommand{\senh}{\mathrm{--ERRO--}}
\providecommand{\arcsen}{\mathrm{--ERRO--}}
\providecommand{\arsenh}{\mathrm{--ERRO--}}

\renewcommand{\sen}[1]{\mathrm{sen}\, \left( #1 \right)}
\renewcommand{\arcsen}[1]{\mathrm{arcsen}\, \left( #1 \right)}
\renewcommand{\senh}[1]{\mathrm{senh}\, \left( #1 \right)}
\renewcommand{\arsenh}[1]{\mathrm{arsenh}\, \left( #1 \right)}

\providecommand{\cos}{\mathrm{--ERRO--}}
\providecommand{\cosh}{\mathrm{--ERRO--}}
\providecommand{\arccos}{\mathrm{--ERRO--}}
\providecommand{\arcosh}{\mathrm{--ERRO--}}

\renewcommand{\cos}[1]{\mathrm{cos}\, \left( #1 \right)}
\renewcommand{\cosh}[1]{\mathrm{cosh}\, \left( #1 \right)}
\renewcommand{\arccos}[1]{\mathrm{arccos}\, \left( #1 \right)}
\renewcommand{\arcosh}[1]{\mathrm{arcosh}\, \left( #1 \right)}

\providecommand{\tg}{\mathrm{--ERRO--}}
\providecommand{\tgh}{\mathrm{--ERRO--}}
\providecommand{\arctg}{\mathrm{--ERRO--}}
\providecommand{\artgh}{\mathrm{--ERRO--}}

\renewcommand{\tg}[1]{\mathrm{tg}\,\left( #1 \right)}
\renewcommand{\tgh}[1]{\mathrm{tgh}\, \left( #1 \right)}
\renewcommand{\arctg}[1]{\mathrm{arctg}\, \left( #1 \right)}
\renewcommand{\artgh}[1]{\mathrm{artg}\, \left( #1 \right)}

\providecommand{\sec}{\mathrm{--ERRO--}}
\providecommand{\cossec}{\mathrm{--ERRO--}}
\providecommand{\cotg}{\mathrm{--ERRO--}}

\renewcommand{\sec}[1]{\mathrm{sec}\, \left( #1 \right)}
\renewcommand{\cossec}[1]{\mathrm{cossec}\, \left( #1 \right)}
\renewcommand{\cotg}[1]{\mathrm{cotg}\, \left( #1 \right)}

\providecommand{\arcsec}{\mathrm{--ERRO--}}
\providecommand{\arccossec}{\mathrm{--ERRO--}}
\providecommand{\arccotg}{\mathrm{--ERRO--}}

\renewcommand{\arcsec}[1]{\mathrm{sec}\, \left( #1 \right)}
\renewcommand{\arccossec}[1]{\mathrm{cossec}\, \left( #1 \right)}
\renewcommand{\arccotg}[1]{\mathrm{cotg}\, \left( #1 \right)}

\providecommand{\sech}{\mathrm{--ERRO--}}
\providecommand{\cossech}{\mathrm{--ERRO--}}
\providecommand{\cotgh}{\mathrm{--ERRO--}}

\renewcommand{\sech}[1]{\mathrm{sech}\, \left( #1 \right)}
\renewcommand{\cossech}[1]{\mathrm{cossech}\, \left( #1 \right)}
\renewcommand{\cotgh}[1]{\mathrm{cotgh}\, \left( #1 \right)}

\providecommand{\arsech}{\mathrm{--ERRO--}}
\providecommand{\arcossech}{\mathrm{--ERRO--}}
\providecommand{\arcotgh}{\mathrm{--ERRO--}}

\renewcommand{\arsech}[1]{\mathrm{arsech}\, \left( #1 \right)}
\renewcommand{\arcossech}[1]{\mathrm{arcossech}\, \left( #1 \right)}
\renewcommand{\arcotgh}[1]{\mathrm{arcotgh}\, \left( #1 \right)}


    \theoremstyle{definition}
    \newtheorem{definition}{Definição}
    
    \theoremstyle{remark}
    \newtheorem*{remark}{Nota}

\begin{document}
    \theoremstyle{definition}
    %Seja $m \in \mathbb{N}$ e seja $A$ uma matriz qualquer.
    %\begin{definition}{Matriz quadrada}
        %Toda matriz de ordem $m \times m$.
    %\end{definition}
    %\begin{definition}{Matriz coluna}
        %Matriz de ordem $m \times 1$
    %\end{definition}
    %\begin{definition}{Matriz linha}
        %Matriz de ordem $1 \times m$
    %\end{definition}
    %\begin{definition}{}
        %
    %\end{definition}
    %\begin{definition}{}
        %
    %\end{definition}
    %\begin{definition}{}
        %
    %\end{definition}
    %\begin{definition}{}
        %
    %\end{definition}
    %\begin{definition}{}
        %
    %\end{definition}
    \begin{enumerate}
        \item $A = B \implies A^T = B^T$
        \item $\left( A^T \right)^T = A$
        \item $(A + B)^T = A^T + B^T$
        \item $(AB)^T = B^T A^T$
        \item $(kA)^T = kA^T$, $k \in \mathbb{R}$
    \end{enumerate}
    \begin{definition}{Traço}
        O traço de uma matriz quadrada $A$ é o somatório dos elementos da diagonal principal. Denota-se $\tr{A}$.
    \end{definition}
    Seja $k \in \mathbb{R}$.
    \begin{enumerate}
        \item $\tr{kA} = k\, \tr{A}$
        \item $\tr{A+B} = \tr{A} + \tr{B}$
        \item $\tr{AB} = \tr{BA}$
        \item $\tr{AB} \neq \tr{A} \tr{B}$
        \item $\tr{A^T} = \tr{A}$
    \end{enumerate}
    \begin{definition}{Matriz simétrica}
        Uma matriz quadrada é simétrica se, e somente se $A = A^T$.
    \end{definition}
    \theoremstyle{remark}
    \begin{remark}
        O produto $AA^T$ é sempre uma matriz simétrica.
    \end{remark}
    \begin{definition}{Matriz antissimétrica}
        Uma matriz é dita antissimétrica se, e somente se $A^T = -A$.
    \end{definition}
    \begin{remark}
        $A$ é uma matriz antissimétrica se, e somente se os elementos posicionados simetricamente com relação à diagonal principal
        são opostos, isto é, $a_{ij} = -a_{ji}$.
    \end{remark}
    \begin{definition}{Matriz ortogonal}
        Uma matriz quadrada é dita ortogonal se, e somente se $A^T = A^{-1}$
    \end{definition}
    \begin{remark}
        Na matriz ortogonal ocorre: $AA^{-1} = I \implies AA^T = I$.
    \end{remark}
    Propriedades dos determinantes:
    \begin{enumerate}
        \item $\det A = \det{A^T}$.
        \item $\det{AB} = \det A \det B$.
        \item $\det{A+B} \neq \det A + \det B$.
        \item Se a matriz $A$ possui uma linha ou coluna nulos então $\det A = 0$.
        \item $\det{kA} = k^n \det A$.
        \item Se A é inversível então $\det A = \dfrac{1}{\det {A^{-1}} }$.
        \item Se $A$ é uma matriz ortogonal ocorre $\det A = \pm 1$.
        \item Se $A$ é uma matriz triangular, então $\det A = a_{11} \cdot a_{22} \cdot a_{33} \hdots a_{nn}$.
    \end{enumerate}
\end{document}
