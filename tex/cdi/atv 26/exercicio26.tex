\documentclass{article}
\usepackage{amsmath}
\usepackage{amssymb}
\usepackage{amsthm}
%\usepackage[portuges]{babel}
\usepackage[utf8]{inputenc}
\usepackage[T1]{fontenc}

\let\oldln\ln
\renewcommand{\ln}[1]{ \oldln {(#1)}}
\newcommand{\abs}[1]{\lvert #1 \rvert}

%\let\oldint\int
%\newcommand{\fint}[3][s][s]{ \int_{#1}^{#2}{\left(#3\right)}}

%\newcommand{\dint}[4]{ \int_{#1}^{#2} {\left(#3\right)}\, \mathrm{d}#4}

\newcommand{\paren}[1]{ \left( #1 \right)}
\newcommand{\colch}[1]{ \left[ #1 \right]}
\newcommand{\tr}[1]{\mathrm{tr} \left( #1 \right)}
\let\olddet\det
\renewcommand{\det}[1]{\olddet{ \left( #1 \right)} }

\providecommand{\sen}{\mathrm{--ERRO--}}
\providecommand{\senh}{\mathrm{--ERRO--}}
\providecommand{\arcsen}{\mathrm{--ERRO--}}
\providecommand{\arsenh}{\mathrm{--ERRO--}}

\renewcommand{\sen}[1]{\mathrm{sen}\, \left( #1 \right)}
\renewcommand{\arcsen}[1]{\mathrm{arcsen}\, \left( #1 \right)}
\renewcommand{\senh}[1]{\mathrm{senh}\, \left( #1 \right)}
\renewcommand{\arsenh}[1]{\mathrm{arsenh}\, \left( #1 \right)}

\providecommand{\cos}{\mathrm{--ERRO--}}
\providecommand{\cosh}{\mathrm{--ERRO--}}
\providecommand{\arccos}{\mathrm{--ERRO--}}
\providecommand{\arcosh}{\mathrm{--ERRO--}}

\renewcommand{\cos}[1]{\mathrm{cos}\, \left( #1 \right)}
\renewcommand{\cosh}[1]{\mathrm{cosh}\, \left( #1 \right)}
\renewcommand{\arccos}[1]{\mathrm{arccos}\, \left( #1 \right)}
\renewcommand{\arcosh}[1]{\mathrm{arcosh}\, \left( #1 \right)}

\providecommand{\tg}{\mathrm{--ERRO--}}
\providecommand{\tgh}{\mathrm{--ERRO--}}
\providecommand{\arctg}{\mathrm{--ERRO--}}
\providecommand{\artgh}{\mathrm{--ERRO--}}

\renewcommand{\tg}[1]{\mathrm{tg}\,\left( #1 \right)}
\renewcommand{\tgh}[1]{\mathrm{tgh}\, \left( #1 \right)}
\renewcommand{\arctg}[1]{\mathrm{arctg}\, \left( #1 \right)}
\renewcommand{\artgh}[1]{\mathrm{artg}\, \left( #1 \right)}

\providecommand{\sec}{\mathrm{--ERRO--}}
\providecommand{\cossec}{\mathrm{--ERRO--}}
\providecommand{\cotg}{\mathrm{--ERRO--}}

\renewcommand{\sec}[1]{\mathrm{sec}\, \left( #1 \right)}
\renewcommand{\cossec}[1]{\mathrm{cossec}\, \left( #1 \right)}
\renewcommand{\cotg}[1]{\mathrm{cotg}\, \left( #1 \right)}

\providecommand{\arcsec}{\mathrm{--ERRO--}}
\providecommand{\arccossec}{\mathrm{--ERRO--}}
\providecommand{\arccotg}{\mathrm{--ERRO--}}

\renewcommand{\arcsec}[1]{\mathrm{sec}\, \left( #1 \right)}
\renewcommand{\arccossec}[1]{\mathrm{cossec}\, \left( #1 \right)}
\renewcommand{\arccotg}[1]{\mathrm{cotg}\, \left( #1 \right)}

\providecommand{\sech}{\mathrm{--ERRO--}}
\providecommand{\cossech}{\mathrm{--ERRO--}}
\providecommand{\cotgh}{\mathrm{--ERRO--}}

\renewcommand{\sech}[1]{\mathrm{sech}\, \left( #1 \right)}
\renewcommand{\cossech}[1]{\mathrm{cossech}\, \left( #1 \right)}
\renewcommand{\cotgh}[1]{\mathrm{cotgh}\, \left( #1 \right)}

\providecommand{\arsech}{\mathrm{--ERRO--}}
\providecommand{\arcossech}{\mathrm{--ERRO--}}
\providecommand{\arcotgh}{\mathrm{--ERRO--}}

\renewcommand{\arsech}[1]{\mathrm{arsech}\, \left( #1 \right)}
\renewcommand{\arcossech}[1]{\mathrm{arcossech}\, \left( #1 \right)}
\renewcommand{\arcotgh}[1]{\mathrm{arcotgh}\, \left( #1 \right)}


\begin{document}
    Mateus Schroeder da Silva - Exercício 26 da Lista\\
    $y = f(x) = x^4-5x^2+4$ e as retas $x=0$ e $x=2$.
    Seja $m=x^2$ temos
    $m^2-5m+4$ Queremos encontrar as raízes:
    $$\Delta = (-5)^2-4 \cdot 1 \cdot 4 = 9 $$
    $$m' = \dfrac{5 \pm 3}{2} = \{ 4,1 \} $$
    $$ m \in \{ 4,1 \} \implies x \in \{ \pm 1, \pm 2 \} $$
    Para nós será relevante somente $x = 1$, pois este é o único local tal que a região à direita e à esquerda nos interessa para o cálculo da área. Precisaremos obter a antiderivada de $ \abs{ f(x) }$. Passando desse ponto a integral terá a tangente igual a $-1$ vezes a tangente que tería no caso de $f(x)$ ao invés de $\abs{f(x)}$

    A área da região é dada por 
    $$ \int_0^2 \abs{f(x)} dx = ( F(2) - F(1) ) + ( F(1) - F(0) )$$ onde 
    $$ F = \int {\abs{x^4 - 5x^2 + 4} } = \dfrac{1}{5} x^5 - \dfrac{5}{3} x^3 + 4x $$ se $x \in (0,1) $
    $$ F = \int {\abs{x^4 - 5x^2 + 4} } = -\left[ \dfrac{1}{5} x^5 - \dfrac{5}{3} x^3 + 4x \right] $$ se $x \in (1,2) $

    $$R_1 = F(1) - F(0) = F(1) = \dfrac{1}{5} - \dfrac{5}{3} + 4 = \dfrac{38}{15} $$
    $$R_2= F(2) - F(1) = \left[ -\dfrac{1}{5} 2^5 + \dfrac{5}{3} 2^3 - 4 \cdot 2 \right] - \left[ - \dfrac{1}{5} +\dfrac{5}{3} - 4 \right] = \dfrac{22}{15}$$
    portanto
    $$R = R_1 + R_2 = \dfrac{38}{15} + \dfrac{22}{15} = \dfrac{60}{15} = 4 $$

\end{document}
