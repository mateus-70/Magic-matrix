\documentclass{article}
\usepackage{amsmath}
\usepackage{amssymb}
\usepackage{amsthm}
%\usepackage[portuges]{babel}
\usepackage[utf8]{inputenc}
\usepackage[T1]{fontenc}

\let\oldln\ln
\renewcommand{\ln}[1]{ \oldln {(#1)}}
\newcommand{\abs}[1]{\lvert #1 \rvert}

%\let\oldint\int
%\newcommand{\fint}[3][s][s]{ \int_{#1}^{#2}{\left(#3\right)}}

%\newcommand{\dint}[4]{ \int_{#1}^{#2} {\left(#3\right)}\, \mathrm{d}#4}

\newcommand{\paren}[1]{ \left( #1 \right)}
\newcommand{\colch}[1]{ \left[ #1 \right]}
\newcommand{\tr}[1]{\mathrm{tr} \left( #1 \right)}
\let\olddet\det
\renewcommand{\det}[1]{\olddet{ \left( #1 \right)} }

\providecommand{\sen}{\mathrm{--ERRO--}}
\providecommand{\senh}{\mathrm{--ERRO--}}
\providecommand{\arcsen}{\mathrm{--ERRO--}}
\providecommand{\arsenh}{\mathrm{--ERRO--}}

\renewcommand{\sen}[1]{\mathrm{sen}\, \left( #1 \right)}
\renewcommand{\arcsen}[1]{\mathrm{arcsen}\, \left( #1 \right)}
\renewcommand{\senh}[1]{\mathrm{senh}\, \left( #1 \right)}
\renewcommand{\arsenh}[1]{\mathrm{arsenh}\, \left( #1 \right)}

\providecommand{\cos}{\mathrm{--ERRO--}}
\providecommand{\cosh}{\mathrm{--ERRO--}}
\providecommand{\arccos}{\mathrm{--ERRO--}}
\providecommand{\arcosh}{\mathrm{--ERRO--}}

\renewcommand{\cos}[1]{\mathrm{cos}\, \left( #1 \right)}
\renewcommand{\cosh}[1]{\mathrm{cosh}\, \left( #1 \right)}
\renewcommand{\arccos}[1]{\mathrm{arccos}\, \left( #1 \right)}
\renewcommand{\arcosh}[1]{\mathrm{arcosh}\, \left( #1 \right)}

\providecommand{\tg}{\mathrm{--ERRO--}}
\providecommand{\tgh}{\mathrm{--ERRO--}}
\providecommand{\arctg}{\mathrm{--ERRO--}}
\providecommand{\artgh}{\mathrm{--ERRO--}}

\renewcommand{\tg}[1]{\mathrm{tg}\,\left( #1 \right)}
\renewcommand{\tgh}[1]{\mathrm{tgh}\, \left( #1 \right)}
\renewcommand{\arctg}[1]{\mathrm{arctg}\, \left( #1 \right)}
\renewcommand{\artgh}[1]{\mathrm{artg}\, \left( #1 \right)}

\providecommand{\sec}{\mathrm{--ERRO--}}
\providecommand{\cossec}{\mathrm{--ERRO--}}
\providecommand{\cotg}{\mathrm{--ERRO--}}

\renewcommand{\sec}[1]{\mathrm{sec}\, \left( #1 \right)}
\renewcommand{\cossec}[1]{\mathrm{cossec}\, \left( #1 \right)}
\renewcommand{\cotg}[1]{\mathrm{cotg}\, \left( #1 \right)}

\providecommand{\arcsec}{\mathrm{--ERRO--}}
\providecommand{\arccossec}{\mathrm{--ERRO--}}
\providecommand{\arccotg}{\mathrm{--ERRO--}}

\renewcommand{\arcsec}[1]{\mathrm{sec}\, \left( #1 \right)}
\renewcommand{\arccossec}[1]{\mathrm{cossec}\, \left( #1 \right)}
\renewcommand{\arccotg}[1]{\mathrm{cotg}\, \left( #1 \right)}

\providecommand{\sech}{\mathrm{--ERRO--}}
\providecommand{\cossech}{\mathrm{--ERRO--}}
\providecommand{\cotgh}{\mathrm{--ERRO--}}

\renewcommand{\sech}[1]{\mathrm{sech}\, \left( #1 \right)}
\renewcommand{\cossech}[1]{\mathrm{cossech}\, \left( #1 \right)}
\renewcommand{\cotgh}[1]{\mathrm{cotgh}\, \left( #1 \right)}

\providecommand{\arsech}{\mathrm{--ERRO--}}
\providecommand{\arcossech}{\mathrm{--ERRO--}}
\providecommand{\arcotgh}{\mathrm{--ERRO--}}

\renewcommand{\arsech}[1]{\mathrm{arsech}\, \left( #1 \right)}
\renewcommand{\arcossech}[1]{\mathrm{arcossech}\, \left( #1 \right)}
\renewcommand{\arcotgh}[1]{\mathrm{arcotgh}\, \left( #1 \right)}


\begin{document}
    Mateus Schroeder da Silva \\
    Exercício da lista: 18 i \\

    $$ \int_1^4 \paren{\dfrac{x}{\sqrt{2+4x} }} dx $$ \\
    $$ F(x) = \int \paren{\dfrac{x}{\sqrt{2+4x}} } dx $$ \\
    $$ u = 2+4x \leadsto du = 4dx $$ \\
    $$ \int \paren{ \dfrac{x}{\sqrt{u} } \dfrac{du}{4} } = \dfrac{1}{4} \int \dfrac{x}{\sqrt{u}} du $$ \\
    Multiplicando por $2/2$ e somando $ 2/\sqrt{u} - 2/\sqrt{u}$
    $$ \dfrac{1}{16} \int {\dfrac{2+4x}{\sqrt{u} } - \dfrac{2}{\sqrt{u}} } du = 
    \dfrac{1}{16} \int { \dfrac{u}{ \sqrt{u} } du } - 2 \int \dfrac{du}{\sqrt{u}}$$ \\
    Como 
    $$ \int { u^{ \frac{1}{2} } } du = \frac{2}{3} u^{ \frac{3}{2} }$$ e \\
    $$ \int { u^{\frac{-1}{2} } = 2u^{\frac{1}{2} }}  $$ \\
    então
    $$ \dfrac{1}{16} \paren{ \int { u^{\frac{1}{2} } du } - 2 \int {u^{\frac{-1}{2}} du} } = \dfrac{1}{16} \paren{\dfrac{2u^\frac{3}{2} }{3} - 2u^\frac{1}{2} }$$ \\
    Temos então que:
    $$ F(x) = \dfrac{\sqrt{ \paren{2+4x}^3 } }{24} - \dfrac{\sqrt{2+4x } }{4} $$ \\
    $$ F(1) = \dfrac{\sqrt{ (2+ 4 \cdot 1) ^3}}{24} - \dfrac{\sqrt{2+4 \cdot 1}}{4} $$ \\
    $$ F(4) = \dfrac{ ( \sqrt{2+ 4 \cdot 4 })^3 }{24} - \dfrac{\sqrt{ 2 + 4 \cdot 4} }{4}$$ \\
    O valor da integral é, portanto
    $$ \int_1^4 \paren{\dfrac{x}{\sqrt{2+4x} }} dx = F(4) - F(1) \approx 2.1213 $$ \\
    $$ $$ \\
    $$ $$ \\
    $$ $$ \\

\end{document}
